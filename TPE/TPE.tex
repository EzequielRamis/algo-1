\documentclass[11pt]{article}

\input{Algo1Macros.tex}
\usepackage{caratula}

\newcommand{\br}{\hfill\break}
\newcommand{\bloque}[2]{
	\ensuremath{
		$\begin{adjustwidth}{#1}{}
			$#2$
		\end{adjustwidth}
		$
	}
}

\begin{document}
%Caratula
\titulo{Trabajo Práctico de Especificación}
\subtitulo{Análisis Habitacional Argentino}
\fecha{08/09/2021}
\materia{Lc. Computación}
\integrante{Simon, Martino}{374/21}{martinosimon@gmail.com}
\integrante{Ramis, Ezequiel}{881/21}{ezequielramis.hello@gmail.com}
\maketitle
%Creación de indice
\tableofcontents
\newpage

\section{Base de datos}

Antes de empezar con la especificación, vamos a definir los siguientes
auxiliares para acceder más declarativamente a los atributos de cada tabla: \\
\br\comentario{Tabla Hogares}\\

\aux{@hogCodusu}{}{\ent}{ord(HOGCODUSU)}

\aux{@hogAño}{}{\ent}{ord(HOGANIO)}

\aux{@hogTrimestre}{}{\ent}{ord(HOGTRIMESTRE)}

\aux{@hogLatitud}{}{\ent}{ord(HOGLATITUD)}

\aux{@hogLongitud}{}{\ent}{ord(HOGLONGITUD)}

\aux{@ii7}{}{\ent}{ord(II7)}

\aux{@region}{}{\ent}{ord(REGION)}

\aux{@mas500}{}{\ent}{ord(MAS 500)}

\aux{@iv1}{}{\ent}{ord(IV1)}

\aux{@iv2}{}{\ent}{ord(IV2)}

\aux{@ii2}{}{\ent}{ord(II2)}

\aux{@ii3}{}{\ent}{ord(II3)}

\br\comentario{Tabla Personas}\\

\aux{@indCodusu}{}{\ent}{ord(INDCODUSU)}

\aux{@componente}{}{\ent}{ord(COMPONENTE)}

\aux{@indAño}{}{\ent}{ord(INDANIO)}

\aux{@indTrimestre}{}{\ent}{ord(INDTRIMESTRE)}

\aux{@ch4}{}{\ent}{ord(CH4)}

\aux{@ch6}{}{\ent}{ord(CH6)}

\aux{@nivelEd}{}{\ent}{ord(NIVEL ED)}

\aux{@estado}{}{\ent}{ord(ESTADO)}

\aux{@catOcup}{}{\ent}{ord(CAT OCUP)}

\aux{@p47t}{}{\ent}{ord(p47T)}

\aux{@pp04g}{}{\ent}{ord(PP04G)}

\section{Especificación}
\subsection{Ejercicio 1}

\begin{proc}{esEncuestaVálida}{\In th: $eph_h$, \In ti: $eph_i$, \Out result: \bool}{}
	\pre{ True }
	\post{ result=true \Iff (
		\bloque{+6em}{
			esMatriz(th) \wedge esMatriz(ti) \wedge\\
			\longitud{th} > 0 \wedge \longitud{ti} > 0 \wedge\\
			esTablaHogarCompleta(th) \wedge esTablaIndividuoCompleta(ti) \wedge\\
			\neg hayIndividuoSinHogar(th, ti) \wedge \neg hayHogarSinIndividuo(th, ti) \wedge\\
			\neg hayIndividuoRepetido(ti) \wedge \neg hayHogarRepetido(th) \wedge\\
			localizacionesValidas(th) \wedge\\
			mismaFechaDeRelevamiento(th, ti) \wedge\\
			todosHogaresConMenosDe21Integrantes(th, ti) \wedge\\
			todosHogaresConMasHabitacionesQueDormitorios(th, ti) \wedge\\
			todosAtributosEnRango(th, ti)
		}
		)}
\end{proc}

\pred{esMatriz}{s: \matriz{dato}}{
	\longitud{s} > 0 \yLuego \neg \existe{t}{\TLista{dato}}{(t\in s) \wedge \longitud{head(s)}\neq\longitud{t}}{}
}

\pred{hayIndividuoRepetido}{s: $eph_i$}{
	\existe{a,b}{individuo}{(a\in s \wedge b\in s) \wedge
		a[@indCodusu] = b[@indCodusu] \wedge
		a[@componente] = b[@componente]}
	{}
}

\pred{hayHogarRepetido}{s: $eph_h$}{
	\existe{a,b}{hogar}{(a\in s \wedge b\in s) \wedge
		a[@hogCodusu] = b[@hogCodusu]}
	{}
}

\end{document}