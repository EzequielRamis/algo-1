\documentclass[11pt]{article}

\input{Algo1Macros.tex}
\usepackage{caratula}
\usepackage[hidelinks]{hyperref}

\newcommand{\br}{\hfill\break}
\newcommand{\bloque}[2]{
	\ensuremath{
		$\begin{adjustwidth}{#1}{}
			$#2$
		\end{adjustwidth}
		$
	}
}

\begin{document}
%Caratula
\titulo{Trabajo Práctico de Especificación}
\subtitulo{Análisis Habitacional Argentino}
\fecha{08/09/2021}
\materia{Lc. Computación}
\integrante{Simon, Martino}{374/21}{martinosimon@gmail.com}
\integrante{Ramis, Ezequiel}{881/21}{ezequielramis.hello@gmail.com}
\maketitle
%Creación de indice
\tableofcontents
\newpage

\section{Auxiliares para base de datos y generales}

Antes de empezar con la especificación de procedimientos, vamos a definir
auxiliares para acceder más declarativamente a los atributos de cada tabla, y
para uso general: \\ \br\comentario{Tabla Hogares}\\

\aux{@hogCodusu}{}{\ent}{ord(HOGCODUSU)}

\aux{@hogAño}{}{\ent}{ord(HOGA\tilde NO)}

\aux{@hogTrimestre}{}{\ent}{ord(HOGTRIMESTRE)}

\aux{@hogLatitud}{}{\ent}{ord(HOGLATITUD)}

\aux{@hogLongitud}{}{\ent}{ord(HOGLONGITUD)}

\aux{@ii7}{}{\ent}{ord(II7)}

\aux{@region}{}{\ent}{ord(REGION)}

\aux{@mas500}{}{\ent}{ord(MAS\_500)}

\aux{@iv1}{}{\ent}{ord(IV1)}

\aux{@iv2}{}{\ent}{ord(IV2)}

\aux{@ii2}{}{\ent}{ord(II2)}

\aux{@ii3}{}{\ent}{ord(II3)}

\br\comentario{Tabla Personas}\\

\aux{@indCodusu}{}{\ent}{ord(INDCODUSU)}

\aux{@componente}{}{\ent}{ord(COMPONENTE)}

\aux{@indAño}{}{\ent}{ord(INDA\tilde NO)}

\aux{@indTrimestre}{}{\ent}{ord(INDTRIMESTRE)}

\aux{@ch4}{}{\ent}{ord(CH4)}

\aux{@ch6}{}{\ent}{ord(CH6)}

\aux{@nivelEd}{}{\ent}{ord(NIVEL ED)}

\aux{@estado}{}{\ent}{ord(ESTADO)}

\aux{@catOcup}{}{\ent}{ord(CAT OCUP)}

\aux{@p47t}{}{\ent}{ord(P47T)}

\aux{@pp04g}{}{\ent}{ord(PP04G)}

\br\comentario{Varios}\\

\aux{\#atributosHogar}{}{\ent}{12}

\aux{\#atributosPersona}{}{\ent}{11}

\aux{\#habitaciones}{h: hogar}{dato}{h[@iv2]}

\aux{\#dormitorios}{h: hogar}{dato}{h[@ii2]}

\aux{\#personasEnHogar}{ti: $eph_i$, h: hogar}{\ent}{
	\sum_{i = 0}^{|ti|-1} \IfThenElse{personaEnHogar(ti[i], h)}{1}{0}
}

\pred{esCasa}{h: hogar}{
	h[@iv1]=1
}

\pred{personaEnHogar}{i: individuo, h: hogar} {
	i[@indCodusu] = h[@hogCodusu]
}

\pred{hogaresConTrimestresIguales}{h: hogar, g: hogar}{
	h[@hogTrimestre] = g[@hogTrimestre]
}

\section{Especificación}
\subsection{Ejercicio 1}

\begin{proc}{esEncuestaVálida}{\In th: $eph_h$, \In ti: $eph_i$, \Out result: \bool}{}
	\pre{ True }
	\post{ result=true \Iff encuestaValida(th, ti) }
\end{proc}

\pred{encuestaValida}{th: $eph_h$, ti: $eph_i$}{
	% Listo
	esMatriz(th) \wedge esMatriz(ti) \wedge\\
	% Listo
	|th| > 0 \wedge |ti| > 0 \yLuego\\
	% Listo
	esTablaCompleta(th, \#atributosHogar) \wedge esTablaCompleta(ti, \#atributosPersona) \yLuego\\
	% Listo
	\neg hayIndividuoSinHogar(th, ti) \wedge \neg hayHogarSinIndividuo(th, ti) \wedge\\
	% Listo
	\neg hayIndividuoRepetido(ti) \wedge \neg hayHogarRepetido(th) \wedge\\
	% Listo
	mismaFechaDeRelevamiento(th, ti) \wedge\\
	% Listo
	todosHogaresConMenosDe21Integrantes(th, ti) \wedge\\
	% Listo
	todosHogaresConMasHabitacionesQueDormitorios(th) \wedge\\
	% Listo
	todosAtributosEnRango(th, ti) \wedge\\
	atributosValidos(th, ti)
}

\pred{esMatriz}{s: \matriz{dato}}{
	|s| > 0 \implicaLuego
	\neg\existe{t}{\TLista{dato}}{
		(t\in s) \wedge |head(s)|\neq |t|
	}
}

\pred{esTablaCompleta}{s: \matriz{dato}, l: \ent}{
	\paraTodo{f}{\TLista{dato}}{
		f\in s \implica |f| = l
	}
}

\pred{hayIndividuoSinHogar}{th: $eph_h$, ti: $eph_i$}{
	\existe{i}{individuo}{
		i\in ti \yLuego \bloque{+1em}{
			\neg\existe{h}{hogar}{
				h\in th \yLuego personaEnHogar(i, h)
			}
		}
	}
}

\pred{hayHogarSinIndividuo}{th: $eph_h$, ti: $eph_i$}{
	\existe{h}{hogar}{
		h\in th \yLuego \bloque{+1em}{
			\neg\existe{i}{individuo}{
				i\in ti \yLuego personaEnHogar(i, h)
			}
		}
	}
}

\pred{hayIndividuoRepetido}{s: $eph_i$}{
	\existe{a,b}{individuo}{(a\in s \wedge b\in s \wedge a\neq b) \yLuego
		a[@indCodusu] = b[@indCodusu] \wedge
		a[@componente] = b[@componente]}
}

\pred{hayHogarRepetido}{s: $eph_h$}{
	\existe{a,b}{hogar}{(a\in s \wedge b\in s \wedge a\neq b) \yLuego
		a[@hogCodusu] = b[@hogCodusu]}
}

\pred{mismaFechaDeRelevamiento}{th: $eph_h$, ti: $eph_i$}{
	(|th| > 0 \wedge |ti| > 0) \implicaLuego
	(\bloque{+2em}{
		\paraTodo{h}{hogar}{
			h\in th \implicaLuego (\bloque{+2em}{
				h[@hogA\tilde no] = head(h)[@hogA\tilde no] \wedge\\
				hogaresConTrimestresIguales(h, head(th))
			})
		} \wedge\\
		\paraTodo{i}{individuo}{
			i\in ti \implicaLuego (\bloque{+2em}{
				i[@indA\tilde no] = head(ti)[@indA\tilde no] \wedge\\
				i[@indTrimestre] = head(ti)[@indTrimestre] \wedge\\
				i[@indA\tilde no] = head(th)[@hogA\tilde no] \wedge\\
				i[@indTrimestre] = head(th)[@hogTrimestre]
			})
		}
	})
}

\pred{todosHogaresConMenosDe21Integrantes}{th: $eph_h$, ti:
	$eph_i$}{
	\paraTodo{h}{hogar}{
		h\in th \implicaLuego 21 > \#personasEnHogar(ti, h)
	}
}

\pred{todosHogaresConMasHabitacionesQueDormitorios}{th: $eph_h$}{
	\paraTodo{h}{hogar}{h\in th \implicaLuego \#habitaciones(h) \geq \#dormitorios(h)}
}

\pred{todosAtributosEnRango}{th: $eph_h$, ti:
	$eph_i$}{
	\paraTodo{h}{hogar}{ h \in th \implicaLuego\\ 1\leq h[@ii7]\leq 3 \wedge\\
		1\leq h[@region]\leq 6 \wedge\\ 0\leq
		h[@mas500]\leq 1\wedge\\ 1\leq
		h[@iv1]\leq 5\wedge\\ 1\leq
		h[@ii3]\leq 2
	}\wedge\\
	\paraTodo {i}{individuo}{ i\in ti \implicaLuego\\ 1\leq i[@ch4]\leq 2 \wedge\\
		0\leq i[@nivelEd]\leq 1 \wedge\\ -1\leq
		i[@estado]\leq 1 \wedge\\ 0\leq
		i[@catOcup]\leq 4 \wedge\\ 1\leq
		i[@pp04g]\leq 10 \wedge \\ -1\leq
		i[@p47t]
	}
}

\pred{atributosValidos}{th: $eph_h$, ti: $eph_i$}{
	\paraTodo{h}{hogar}{ h \in th \implicaLuego\\ 0 < h[@hogCodusu] \wedge\\ 1800\leq
		h[@hogA \tilde n o] \wedge \\ 1\leq
		h[@hogTrimestre]\leq 4 \wedge \\ 1\leq \#habitaciones(h) \wedge
		\\ 1\leq \#dormitorios(h)
	}\wedge\\
	\paraTodo {i}{individuo}{ i \in ti \implicaLuego\\ 0 < i[@indCodusu] \wedge \\ 1\leq
		i[@componente] \wedge \\ 1800\leq
		i[@indA\tilde n o] \wedge \\ 1\leq
		i[@indTrimestre]\leq 4 \wedge \\ 0\leq
		i[@ch6]
	}
}

\subsection{Ejercicio 2}

\begin{proc}{histHabitacional}{\In th: $eph_h$, \In ti: $eph_i$, \In region: \ent, \Out res: \TLista{\ent}}{}
	\pre{ encuestaValida(th, ti) \wedge 1\leq region\leq 6 }
	\post{ longitudDeHistograma(th, region, res) \wedge
		\bloque{+6em}{
			(\forall{i: \ent})
			(0\leq i < |res| \implicaLuego res[i] =	\#casaConNHabitacionesEnRegion(th, i + 1, region))
		}
	}
\end{proc}

\aux{\#casaConNHabitacionesEnRegion}{th: $eph_h$, n: \ent, r: dato}{\ent}{
	\bloque{+2em}{
		\sum_{i=0}^{|th|-1}\IfThenElse{esCasa(th[i]) \wedge
			\#habitaciones(th[i]) = n \wedge
			th[i][@region]=r}{1}{0}
	}
}

\pred{longitudDeHistograma}{th: $eph_h$, r: \ent, res: \TLista{\ent}}{
	\existe{h}{hogar}{
		h\in th \yLuego esCasa(h) \wedge h[@region]=r \wedge\\
		\paraTodo{g}{hogar}{
			(g\in th \yLuego esCasa(g) \wedge g[@region]=r) \implica
			\#habitaciones(h)\geq \#habitaciones(g)
		} \wedge\\ \#habitaciones(h) = |res|}
}

\subsection{Ejercicio 3}

\begin{proc}{laCasaEstaQuedandoChica}{\In th: $eph_h$, \In ti: $eph_i$, \Out res: \TLista{\float}}{}
	\pre{ encuestaValida(th, ti)}
	\post{ |res| = 6 \yLuego
		\bloque{+6em}{
			\paraTodo{i}{\ent}{
				0 \leq i < |res| \implicaLuego ( \bloque{+2em}{
					(\#casasPorRegion(th, i + 1) \neq 0\implica\bloque{+2em}{res[i] = \#casasCriticasPorRegion(th, ti, i + 1) / \#casasPorRegion(th, i + 1)}) \wedge\\
					(\#casasPorRegion(th, i + 1) = 0 \implica res[i] = 0)
				})
			}
		}
	}
\end{proc}

\aux{\#casasPorRegion}{th: $eph_h$, r: \ent}{\ent}{
	\bloque{+2em}{\sum_{i=0}^{|th|-1}\IfThenElse{esCasa(th[i]) \wedge th[i][@region] =
			r \wedge \neg estaEnGranAglomerado(th[i])}{1}{0}}
}

\aux{\#casasCriticasPorRegion}{th: $eph_h$, ti: $eph_i$, r: \ent}{\ent}{
	\bloque{+2em}{
		\sum_{i=0}^{|th|-1}\IfThenElse{esCasaCritica(th[i], ti) \wedge th[i][@region] = r}{1}{0}
	}
}

\pred{esCasaCritica}{h: hogar, ti: $eph_i$}{
	esCasa(h) \wedge esHacinamientoCritico(h, ti) \wedge \neg estaEnGranAglomerado(h)
}

\pred{esHacinamientoCritico}{h: hogar, ti: $eph_i$}{
	\#personasEnHogar(ti, h) / \#dormitorios(h) > 3
}

\pred{estaEnGranAglomerado}{h: hogar}{h[@mas500] = 1}

\subsection{Ejercicio 4}

\begin{proc}{creceElTeleworkingEnCiudadesGrandes}{\In t1h: $eph_h$, \In t1i: $eph_i$, \In t2h: $eph_h$, \In t2i: $eph_i$, \Out res: \bool}{}
	\pre{ (encuestaValida(t1h, t1i) \wedge encuestaValida(t2h, t2i)) \yLuego (
		\bloque{+6em}{
			esHogarConA\tilde noPrevio(head(t1h), head(t2h)) \wedge\\
			hogaresConTrimestresIguales(head(t1h), head(t2h))
		})
	}
	\post{ res = true \Iff teleworkingEnCiudadesGrandes(t1h, t1i) < teleworkingEnCiudadesGrandes(t2h,t2i) }
\end{proc}

\pred{esHogarConAñoPrevio}{h: hogar, g: hogar}{
	h[@hogA\tilde no] < g[@hogA\tilde no]
}

\aux{teleworkingEnCiudadesGrandes}{th: $eph_h$, ti: $eph_i$}{\float}{
	\bloque{+1em}{
		\IfThenElse{\#personasTrabajandoEnCiudadesGrandes(th, ti) \neq 0}{\bloque{+0em}{\#personasTrabajandoADistanciaEnCiudadesGrandes(th, ti)/\#personasTrabajandoEnCiudadesGrandes(th, ti)}}{0}
	}
}

\aux{\#personasTrabajandoADistanciaEnCiudadesGrandes}{th: $eph_h$, ti: $eph_i$}{\ent}{
	\bloque{+2em}{
		\sum_{i=0}^{|ti|-1}\IfThenElse{viveEnHogarAptoParaTeleworking(ti[i], th) \wedge trabajaADistancia(ti[i])}{1}{0}
	}
}

\aux{\#personasTrabajandoEnCiudadesGrandes}{th: $eph_h$, ti: $eph_i$}{\ent}{
	\bloque{+2em}{
		\sum_{i=0}^{|ti|-1}\IfThenElse{viveEnCiudadGrande(ti[i], th) \wedge trabaja(ti[i])}{1}{0}
	}
}

\pred{viveEnHogarAptoParaTeleworking}{i: individuo, th: $eph_h$}{
	\existe{h}{hogar}{
		h\in th \yLuego\\ personaEnHogar(i, h)  \wedge esCasaODepto(h) \wedge h[@mas500] = 1 \wedge
		h[@ii3] = 1
	}
}

\pred{esCasaODepto}{h: $hogar$}{
	h[@iv1] = 1 \vee h[@iv1] = 2
}

\pred{trabajaADistancia}{i: individuo}{
	i[@pp04g]=6 \wedge trabaja(i)
}

\pred{trabaja}{i: individuo}{
	i[@estado] = 1
}

\pred{viveEnCiudadGrande}{i: individuo, th: $eph_h$}{
	\existe{h}{hogar}{
		h\in th \yLuego\\ personaEnHogar(i, h) \wedge esCasaODepto(h) \wedge h[@mas500] = 1
	}
}

\subsection{Ejercicio 5}

\begin{proc}{costoSubsidioMejora}{\In th: $eph_h$, \In ti: $eph_i$, \In monto: \ent, \Out res: \ent}{}
	\pre{ encuestaValida(th, ti) \wedge monto > 0 }
	\post{ res = \sum_{i=0}^{|th|-1}\IfThenElse{esHogarSubsidiado(th[i], ti)}{monto}{0} }
\end{proc}

\pred{esHogarSubsidiado}{h: hogar, ti: $eph_i$}{
	esCasa(h) \wedge h[@ii7] = 1 \wedge \#dormitorios(h) < \#personasEnHogar(ti,h) - 2
}

\subsection{Ejercicio 6}

\begin{proc}{generarJoin}{\In th: $eph_h$, \In ti: $eph_i$, \Out junta: $joinHI$}{}
	\pre{ encuestaValida(th, ti) }
	\post{
		(\forall hi: hogar\times individuo)(
		\bloque{+6em}{hi\in junta \implica (\bloque{+2em}{
				hi_0\in th \wedge hi_1\in ti \yLuego hi_0[@hogCodusu] = hi_1[@indCodusu]
			})}
	})
\end{proc}

\subsection{Ejercicio 7}

\begin{proc}{ordenarRegionYTipo}{\Inout th: $eph_h$, \Inout ti: $eph_i$}{}
	\pre{ encuestaValida(th, ti) \wedge th = TH \wedge ti = TI }
	\post{
		\bloque{+6em}{
			|th| = |TH| \wedge |ti| = |TI| \yLuego\\
			(\forall h: hogar)(h\in th \Iff h\in TH) \wedge\\
			(\forall i: individuo)(i\in ti \Iff i\in TI) \wedge\\
			hogaresOrdenados(th) \wedge individuosOrdenados(th, ti)
		}
	}
\end{proc}

\pred{hogaresOrdenados}{th: $eph_h$}{
	(\forall i: \ent)
	(0\leq i< |th| -1 \implicaLuego (\bloque{+2em}{
		th[i][@region]\leq th[i+1][@region] \wedge\\
		(th[i][@region] = th[i+1][@region] \implica th[i][@hogCodusu] < th[i+1][@hogCodusu])
	})
}

\pred{individuosOrdenados}{th: $eph_h$, ti:
	$eph_i$}{
	(\forall i: \ent)
	(0\leq i< |ti| -1 \implicaLuego
	\bloque{+2em}{
		indicePorCodusu(th, ti[i][@indCodusu]) = indicePorCodusu(th, ti[i+1][@indCodusu])\implica (
		\bloque{+2em}{ti[i][@indCodusu] = ti[i+1][@indCodusu] \wedge
			ti[i][@componente]<ti[i+1][@componente]})\wedge\\
		indicePorCodusu(th, ti[i][@indCodusu]) > indicePorCodusu(th, ti[i+1][@indCodusu])\implica (
		\bloque{+2em}{ti[i][@indCodusu] > ti[i+1][@indCodusu]})\wedge\\
		indicePorCodusu(th, ti[i][@indCodusu]) < indicePorCodusu(th, ti[i+1][@indCodusu])\implica (
		\bloque{+2em}{ti[i][@indCodusu] < ti[i+1][@indCodusu]})
	})
}

/* Asumo que la lista tiene elementos únicos, y que el elemento pertenece a la
lista */\\ \aux{indicePorCodusu}{s: $eph_h$, e: $dato$}{\ent}{
	\sum_{i=0}^{|s|-1}\IfThenElse{s[i][@hogCodusu]=e}{i}{0}
}

\subsection{Ejercicio 8}

\begin{proc}{muestraHomogenea}{\In th: $eph_h$, \In ti: $eph_i$, \Out res: \TLista{hogar}}{}
	\pre{ encuestaValida(th, ti) }
	\post{  \existe{s}{\TLista{hogar}}{
			\bloque{+6em}{
				esHomogenea(s, th, ti) \wedge\\
				(\forall t: \TLista{hogar})(esHomogenea(t, th, ti) \implica |s| \geq |t|) \wedge (
				\bloque{+2em}{
					(
					|s| < 3 \implica res = <>
					) \wedge\\
					(
					|s| \geq 3 \implica res = s
					)
				})
			}}
	}
\end{proc}

\pred{esHomogenea}{s: \TLista{hogar}, th: $eph_h$, ti: $eph_i$}{
	mismaDiferencia(s) \wedge ordenadaPorIngresos(s,ti) \wedge (\forall h: $hogar$)(h\in s \implica h\in th)
}

\pred{mismaDiferencia}{s: \TLista{hogar}}{
	(\forall i: \ent) (1\leq i < |s| - 1 \implicaLuego diferenciaDeIngresos(s[i], s[i-1], ti) = diferenciaDeIngresos(s[i], s[i+1], ti))
}

\pred{ordenadaPorIngresos}{s: \TLista{hogar}, ti: $eph_i$} {
	(\forall i: \ent)(
	0\leq i < |s| - 1 \implicaLuego
	ingresosEnHogar(s[i], ti) \leq ingresosEnHogar(s[i+1], ti)
	)
}

\aux{diferenciaDeIngresos}{a,b: $hogar$, ti: $eph_i$}{\ent}{
	|ingresosEnHogar(a, ti) - ingresosEnHogar(b, ti)|
}

\aux{ingresosEnHogar}{h: $hogar$, ti: $eph_i$}{\ent}{
	\bloque{+2em}{
		\sum_{i=0}^{|ti|-1}\IfThenElse{personaEnHogar(ti[i], h) \wedge ti[i][@p47T] \neq -1}{ti[i][@p47T]}{0}
	}
}

\subsection{Ejercicio 9}

\begin{proc}{corregirRegion}{\Inout th: $eph_h$, \In ti: $eph_i$}{}
	\pre{ encuestaValida(th, ti) \wedge th = TH }
	\post{ |th|=|TH| \yLuego (\forall i: \ent)(
	0\leq i < |th| \implicaLuego \bloque{+6em}{ (TH[i][@region]=1
	\implica \bloque{+2em}{ th[i][@region]=5 \wedge\\
	th[i][@hogCodusu]=TH[i][@hogCodusu] \wedge\\
	th[i][@hogA\tilde no]=TH[i][@hogA\tilde no] \wedge\\
	th[i][@hogTrimestre]=TH[i][@hogTrimestre] \wedge\\
	th[i][@hogLatitud]=TH[i][@hogLatitud] \wedge\\
	th[i][@hogLongitud]=TH[i][@hogLongitud] \wedge\\
	th[i][@ii7]=TH[i][@ii7] \wedge\\
	th[i][@mas500]=TH[i][@mas500] \wedge\\
	th[i][iv1]=TH[i][iv1] \wedge\\
	th[i][@iv2]=TH[i][@iv2] \wedge\\
	th[i][@ii2]=TH[i][@ii2] \wedge\\
	th[i][@ii3]=TH[i][@ii3] } )\wedge\\
	(TH[i][@region]\neq1 \implica
	th[i]=TH[i]) } ) }
\end{proc}

\subsection{Ejercicio 10}

\begin{proc}{histogramaDeAnillosConcentricos}{\In th: $eph_h$, \In centro: \ent\times\ent, \In distancias: \TLista{\ent}, \Out result: \TLista{\ent}}{}
	\pre{\bloque{+6em}{|distancias| > 0 \wedge\\
			(\forall i: \ent)(0\leq i < |distancias| - 1 \implicaLuego 0 < distancias[i] < distancias[i+1])}}
	\post{ |result| = |distancias| \yLuego
		\bloque{+6em}{
			\paraTodo{i}{\ent}{
				0\leq i < |result| \implicaLuego (\bloque{+2em}{
					(i=0\implica res[i] = hogaresEnRadio(th, centro, -1, distancias[0]))\wedge\\
					(i \neq 0\implica res[i] = hogaresEnRadio(th, centro, distancias[i-1], distancias[i]))} )
			}
		}
	}
\end{proc}

\aux{hogaresEnRadio}{th: $eph_h$, centro: \ent\times\ent,
	desde:\ent ,hasta:\ent}{\ent}{
	\bloque{+2em}{\sum_{j=0}^{|th|-1}\IfThenElse{desde<distanciaEuclidiana(cent ro,(th[j][@hogLatitud], th[j][@hogLongitud])) \leq hasta}{1}{0}}
}

\aux{distanciaEuclidiana}{c1,c2: \ent\times\ent}{\float}{
	\sqrt{(c2_0-c1_0)^2+(c2_1-c1_1)^2}
}

\subsection{Ejercicio 11}

\begin{proc}{quitarIndividuos}{\Inout th: $eph_h$, \Inout ti: $eph_i$, \In busqueda: \TLista{ItemIndividuo\times dato}, \Out result: $eph_h$\times$eph_i$}{}
	\pre{ encuestaValida(th, ti) \wedge busquedaValida(busqueda) \wedge th = TH \wedge ti = TI }
	\post{ 0\leq |th|\leq |TH| \wedge 0\leq |ti|\leq |TI| \wedge |result| = |TI| - |ti| \yLuego
		\bloque{+6em}{
			\paraTodo{h}{hogar}{(h \in TH\wedge hogarLlenoDeCoincidentes(TI, h, busqueda))\Iff(h\in result_0 \wedge h \notin th)} \wedge \\
			\paraTodo{i}{individuo}{(i \in TI\wedge coincideConTerminos(i, busqueda))\Iff(i\in result_1 \wedge i \notin ti)}
		}
	}
\end{proc}

\aux{\#personasEnHogarCoincidentes}{ti: $eph_i$, h: $hogar$, busqueda: \TLista{ItemIndividuo\times dato}}{\ent}{
	\bloque{+2em}{\sum_{j=0}^{|ti|-1} \IfThenElse{personaEnHogar(ti[j], h)\wedge
			coincideConTerminos(ti[j], busqueda)}{1}{0}}
}

\pred{hogarLlenoDeCoincidentes}{ti: $eph_i$, h: $hogar$, busqueda: \TLista{ItemIndividuo\times dato}}{
	\#personasEnHogar(ti,h) = \#personasEnHogarCoincidentes(ti,h,busqueda)
}

\pred{busquedaValida}{busqueda: \TLista{ItemIndividuo\times dato}}{
	\paraTodo{b}{ItemIndividuo\times dato}{b\in busqueda \implica def(ord(b_0))} \yLuego\\
	\neg\existe{i,j}{\ent}{0\leq i,j < |busqueda| \wedge i\neq j \yLuego
		ord(busqueda[i]_0)=ord(busqueda[j]_0)}{}
}

\pred{coincideConTerminos}{i : individuo, busqueda: \TLista{ItemIndividuo\times dato} }{
	\paraTodo{t}{\ent}{0\leq t < |busqueda| \implicaLuego i[ord(busqueda[t]_0)] =
		busqueda[t]_1}
}

\end{document}