\documentclass[11pt]{article}

\input{Algo1Macros.tex}
\usepackage{caratula}
\usepackage[hidelinks]{hyperref}

\newcommand{\br}{\hfill\break}
\newcommand{\bloque}[2]{
	\ensuremath{
		$\begin{adjustwidth}{#1}{}
			$#2$
		\end{adjustwidth}
		$
	}
}

\begin{document}
%Caratula
\titulo{Trabajo Práctico de Especificación}
\subtitulo{Análisis Habitacional Argentino}
\fecha{08/09/2021}
\materia{Lc. Computación}
\integrante{Simon, Martino}{374/21}{martinosimon@gmail.com}
\integrante{Ramis, Ezequiel}{881/21}{ezequielramis.hello@gmail.com}
\maketitle
%Creación de indice
\tableofcontents
\newpage

\section{Auxiliares para base de datos y generales}

Antes de empezar con la especificación de procedimientos, vamos a definir
auxiliares para acceder más declarativamente a los atributos de cada tabla, y
para uso general: \\ \br\comentario{Tabla Hogares}\\

\aux{@hogCodusu}{}{\ent}{ord(HOGCODUSU)}

\aux{@hogAño}{}{\ent}{ord(HOGA\tilde NO)}

\aux{@hogTrimestre}{}{\ent}{ord(HOGTRIMESTRE)}

\aux{@hogLatitud}{}{\ent}{ord(HOGLATITUD)}

\aux{@hogLongitud}{}{\ent}{ord(HOGLONGITUD)}

\aux{@ii7}{}{\ent}{ord(II7)}

\aux{@region}{}{\ent}{ord(REGION)}

\aux{@mas500}{}{\ent}{ord(MAS\_500)}

\aux{@iv1}{}{\ent}{ord(IV1)}

\aux{@iv2}{}{\ent}{ord(IV2)}

\aux{@ii2}{}{\ent}{ord(II2)}

\aux{@ii3}{}{\ent}{ord(II3)}

\br\comentario{Tabla Personas}\\

\aux{@indCodusu}{}{\ent}{ord(INDCODUSU)}

\aux{@componente}{}{\ent}{ord(COMPONENTE)}

\aux{@indAño}{}{\ent}{ord(INDA\tilde NO)}

\aux{@indTrimestre}{}{\ent}{ord(INDTRIMESTRE)}

\aux{@ch4}{}{\ent}{ord(CH4)}

\aux{@ch6}{}{\ent}{ord(CH6)}

\aux{@nivelEd}{}{\ent}{ord(NIVEL ED)}

\aux{@estado}{}{\ent}{ord(ESTADO)}

\aux{@catOcup}{}{\ent}{ord(CAT OCUP)}

\aux{@p47t}{}{\ent}{ord(P47T)}

\aux{@pp04g}{}{\ent}{ord(PP04G)}

\br\comentario{Varios}\\

\aux{\#atributosHogar}{}{\ent}{12}

\aux{\#atributosPersona}{}{\ent}{11}

\aux{\#habitaciones}{h: hogar}{dato}{h[@iv2]}

\aux{\#dormitorios}{h: hogar}{dato}{h[@ii2]}

\aux{\#personasEnHogar}{ti: $eph_i$, h: hogar}{\ent}{
	\sum_{i\in ti} \IfThenElse{personaEnHogar(i, h)}{1}{0}
}

\pred{esCasa}{h: hogar}{
	h[@iv1]=1
}

\pred{personaEnHogar}{i: individuo, h: hogar} {
	i[@indCodusu] = h[@hogCodusu]
}

\pred{hogaresConTrimestresIguales}{h: hogar, g: hogar}{
	h[@hogTrimestre] = g[@hogTrimestre]
}

\section{Especificación}
\subsection{Ejercicio 1}

\begin{proc}{esEncuestaVálida}{\In th: $eph_h$, \In ti: $eph_i$, \Out result: \bool}{}
	\pre{ True }
	\post{ result=true \Iff encuestaValida(th, ti) }
\end{proc}

\pred{encuestaValida}{th: $eph_h$, ti: $eph_i$}{
	% Listo
	esMatriz(th) \wedge esMatriz(ti) \wedge\\
	% Listo
	|th| > 0 \wedge |ti| > 0 \wedge\\
	% Listo
	esTablaCompleta(th, \#atributosHogar) \wedge esTablaCompleta(ti, \#atributosPersona) \wedge\\
	% Listo
	\neg hayIndividuoSinHogar(th, ti) \wedge \neg hayHogarSinIndividuo(th, ti) \wedge\\
	% Listo
	\neg hayIndividuoRepetido(ti) \wedge \neg hayHogarRepetido(th) \wedge\\
	% Listo
	mismaFechaDeRelevamiento(th, ti) \wedge\\
	% Listo
	todosHogaresConMenosDe21Integrantes(th, ti) \wedge\\
	% Listo
	todosHogaresConMasHabitacionesQueDormitorios(th) \wedge\\
	% Listo
	todosAtributosEnRango(th, ti) \wedge\\
	atributosValidos(th, ti)
}

\pred{esMatriz}{s: \matriz{dato}}{
	|s| > 0 \implicaLuego
	\neg\existe{t}{\TLista{dato}}{
		(t\in s) \wedge |head(s)|\neq |t|
	}
}

\pred{esTablaCompleta}{s: \matriz{dato}, l: \ent}{
	\paraTodo{f}{\TLista{dato}}{
		f\in s \implica |f| = l
	}
}

\pred{hayIndividuoSinHogar}{th: $eph_h$, ti: $eph_i$}{
	\existe{i}{individuo}{
		i\in ti \wedge \bloque{+1em}{
			\neg\existe{h}{hogar}{
				h\in th \wedge personaEnHogar(i, h)
			}
		}
	}
}

\pred{hayHogarSinIndividuo}{th: $eph_h$, ti: $eph_i$}{
	\existe{h}{hogar}{
		h\in th \wedge \bloque{+1em}{
			\neg\existe{i}{individuo}{
				i\in ti \wedge personaEnHogar(i, h)
			}
		}
	}
}

\pred{hayIndividuoRepetido}{s: $eph_i$}{
	\existe{a,b}{individuo}{(a\in s \wedge b\in s) \wedge
		a[@indCodusu] = b[@indCodusu] \wedge
		a[@componente] = b[@componente]}
}

\pred{hayHogarRepetido}{s: $eph_h$}{
	\existe{a,b}{hogar}{(a\in s \wedge b\in s) \wedge
		a[@hogCodusu] = b[@hogCodusu]}
}

\pred{mismaFechaDeRelevamiento}{th: $eph_h$, ti: $eph_i$}{
	(|th| > 0 \wedge |ti| > 0) \implicaLuego
	(\bloque{+2em}{
		\paraTodo{h}{hogar}{
			h\in th \implica (\bloque{+2em}{
				h[@hogA\tilde no] = head(h)[@hogA\tilde no] \wedge\\
				hogaresConTrimestresIguales(h, head(th))
			})
		} \wedge\\
		\paraTodo{i}{individuo}{
			i\in ti \implica (\bloque{+2em}{
				i[@indA\tilde n o] = head(th)[@hogA\tilde no] \wedge\\
				i[@indTrimestre] = head(th)[@hogTrimestre]
			})
		}
	})
}

\pred{todosHogaresConMenosDe21Integrantes}{th: $eph_h$, ti:
	$eph_i$}{
	\paraTodo{h}{hogar}{
		h\in th \implica 21 > \#personasEnHogar(ti, h)
	}
}

\pred{todosHogaresConMasHabitacionesQueDormitorios}{th: $eph_h$}{
	\paraTodo{h}{hogar}{h\in th \implica \#habitaciones(h) \geq \#dormitorios(h)}
}

\pred{todosAtributosEnRango}{th: $eph_h$, ti:
	$eph_i$}{
	\paraTodo{h}{hogar}{
		1\leq h[@ii7]\leq 3 \wedge\\ 1\leq
		h[@region]\leq 6 \wedge\\ 0\leq
		h[@mas500]\leq 1\wedge\\ 1\leq
		h[@iv1]\leq 5\wedge\\ 1\leq
		h[@ii3]\leq 2
	}\wedge\\
	\paraTodo {i}{individuo}{
		1\leq i[@ch4]\leq 2 \wedge\\ 0\leq
		i[@nivelEd]\leq 1 \wedge\\ -1\leq
		i[@estado]\leq 1 \wedge\\ 0\leq
		i[@catOcup]\leq 4 \wedge\\ 1\leq
		i[@pp04g]\leq 10 \wedge \\ -1\leq
		i[@p47t]
	}
}

\pred{atributosValidos}{th: $eph_h$, ti: $eph_i$}{
	\paraTodo{h}{hogar}{
		1800\leq h[@hogA \tilde n o] \wedge \\ 1\leq
		h[@hogTrimestre]\leq 4 \wedge \\ 1\leq \#habitaciones(h) \wedge
		\\ 1\leq \#dormitorios(h)
	}\wedge\\
	\paraTodo {i}{individuo}{
		1\leq i[@componente] \wedge \\ 1800\leq
		i[@indA\tilde n o] \wedge \\ 1\leq
		i[@indTrimestre]\leq 4 \wedge \\ 0\leq
		i[@ch6]
	}
}

\subsection{Ejercicio 2}

\begin{proc}{histHabitacional}{\In th: $eph_h$, \In ti: $eph_i$, \In region: \ent, \Out res: \TLista{\ent}}{}
	\pre{ encuestaValida(th, ti) \wedge 1\leq region\leq 6 \wedge (\exists{h: hogar})(h\in th \wedge h[@region] = region \wedge esCasa(h))}
	\post{ longitudDeHistograma(th, res) \yLuego
		\bloque{+6em}{
			(\forall{i: \ent})
			(0\leq i < |res| \implicaLuego res[i] =	\#casaConNHabitacionesEnRegion(th, i + 1, region))
		}
	}
\end{proc}

\aux{\#casaConNHabitacionesEnRegion}{th: $eph_h$, n: \ent, r: dato}{\ent}{
	\bloque{+2em}{
		\sum_{h\in th}\IfThenElse{esCasa(h) \wedge \#habitaciones(h) = n \wedge h[@region]=r}{1}{0}
	}
}

\pred{longitudDeHistograma}{th: $eph_h$, res: \TLista{\ent}}{
	\existe{h}{hogar}{
		h\in th \wedge esCasa(h) \wedge\\ \paraTodo{g}{hogar}{
			(g\in th \wedge esCasa(g)) \implica \#habitaciones(h)\geq \#habitaciones(g)
		} \wedge\\
		\#habitaciones(h) = |res|}
}

\subsection{Ejercicio 3}

\begin{proc}{laCasaEstaQuedandoChica}{\In th: $eph_h$, \In ti: $eph_i$, \Out res: \TLista{\float}}{}
	\pre{ encuestaValida(th, ti) \yLuego (\forall{reg: \ent})(1\leq reg\leq 6 \implicaLuego\#casasPorRegion(th, reg)>0)}
	\post{ |res| = 6 \yLuego
		\bloque{+6em}{
			\paraTodo{i}{\ent}{
				0 \leq i < |res| \implicaLuego ( \bloque{+2em}{
					0\leq res[i]\leq 1 \wedge\\ res[i] = \#casasCriticasPorRegion(th, ti, i + 1) / \#casasPorRegion(th, i + 1)
				})
			}
		}
	}
\end{proc}

\aux{\#casasPorRegion}{th: $eph_h$, r: \ent}{\ent}{
	\sum_{h\in th}\IfThenElse{esCasa(h) \wedge h[@region] = r}{1}{0}
}

\aux{\#casasCriticasPorRegion}{th: $eph_h$, ti: $eph_i$, r: \ent}{\ent}{
	\bloque{+2em}{
		\sum_{h\in th}\IfThenElse{esCasaCritica(h, ti) \wedge h[@region] = r}{1}{0}
	}
}

\pred{esCasaCritica}{h: hogar, ti: $eph_i$}{
	esCasa(h) \wedge esHacinamientoCritico(h, ti) \wedge \neg estaEnGranAglomerado(h)
}

\pred{esHacinamientoCritico}{h: hogar, ti: $eph_i$}{
	\#personasEnHogar(ti, h) / \#dormitorios(h) > 3
}

\pred{estaEnGranAglomerado}{h: hogar}{h[@mas500] = 1}

\subsection{Ejercicio 4}

\begin{proc}{creceElTeleworkingEnCiudadesGrandes}{\In t1h: $eph_h$, \In t1i: $eph_i$, \In t2h: $eph_h$, \In t2i: $eph_i$, \Out res: \bool}{}
	\pre{ (encuestaValida(t1h, t1i) \wedge encuestaValida(t2h, t2i)) \yLuego (
		\bloque{+6em}{
			esHogarConA\tilde noPrevio(head(t1h), head(t2h)) \wedge\\
			hogaresConTrimestresIguales(head(t1h), head(t2h))
		})
	}
	\post{ res = true \Iff teleworkingEnCiudadesGrandes(t1h, t1i) < teleworkingEnCiudadesGrandes(t2h,t2i) }
\end{proc}

\pred{esHogarConAñoPrevio}{h: hogar, g: hogar}{
	h[@hogA\tilde no] < g[@hogA\tilde no]
}

\aux{teleworkingEnCiudadesGrandes}{th: $eph_h$, ti: $eph_i$}{\float}{
	\bloque{+2em}{
		\#personasTrabajandoADistanciaEnCiudadesGrandes(th, ti)/\#personasTrabajandoEnCiudadesGrandes(th, ti)
	}
}

\aux{\#personasTrabajandoADistanciaEnCiudadesGrandes}{th: $eph_h$, ti: $eph_i$}{\ent}{
	\bloque{+2em}{
		\sum_{i\in ti}\IfThenElse{viveEnHogarAptoParaTeleworking(i, th) \wedge trabajaADistancia(i)}{1}{0}
	}
}

\aux{\#personasTrabajandoEnCiudadesGrandes}{th: $eph_h$, ti: $eph_i$}{\ent}{
	\bloque{+2em}{
		\sum_{i\in ti}\IfThenElse{viveEnCiudadGrande(i, th) \wedge trabaja(i)}{1}{0}
	}
}

\pred{viveEnHogarAptoParaTeleworking}{i: individuo, th: $eph_h$}{
	\existe{h}{hogar}{
		h\in th \wedge\\ personaEnHogar(i, h) \wedge h[@mas500] = 1 \wedge
		h[@ii3] = 1 \wedge (h[@iv1] = 1 \vee h[@iv1] = 2)
	}
}

\pred{trabajaADistancia}{i: individuo}{
	i[@pp04g]=6 \wedge trabaja(i)
}

\pred{trabaja}{i: individuo}{
	i[@estado] = 1
}

\pred{viveEnCiudadGrande}{i: individuo, th: $eph_h$}{
	\existe{h}{hogar}{
		h\in th \wedge\\ personaEnHogar(i, h) \wedge h[@mas500] = 1
	}
}

\subsection{Ejercicio 5}

\begin{proc}{costoSubsidioMejora}{\In th: $eph_h$, \In ti: $eph_i$, \In monto: \ent, \Out res: \ent}{}
	\pre{ encuestaValida(th, ti) \wedge monto > 0 }
	\post{ res = \sum_{h\in th}\IfThenElse{esHogarSubsidiado(h, ti)}{monto}{0} }
\end{proc}

\pred{esHogarSubsidiado}{h: hogar, ti: $eph_i$}{
	esCasa(h) \wedge h[@ii7] = 1 \wedge \#dormitorios(h) < \#personasEnHogar(ti,h) - 2
}
\end{document}

