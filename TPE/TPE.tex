\documentclass[11pt]{article}

\input{Algo1Macros.tex}
\usepackage{caratula}

\begin{document}
%Caratula
\titulo{Trabajo Práctico de Especificación}
\fecha{08/09/2021}
\materia{Lc. Computación}
\integrante{Simon, Martino}{374/21}{martinosimon@gmail.com}
\integrante{Ramis, Ezequiel}{881/21}{ezequielramis.hello@gmail.com}
\maketitle
%Creación de indice
\tableofcontents
\newpage

\section{Acceso a base de datos}

Antes de empezar con la especificación, vamos a definir los siguientes
auxiliares para acceder más declarativamente a los atributos de cada tabla:

\comentario{Tabla Hogares}

\aux{@hogCodusu}{}{\ent}{ord(HOGCODUSU)}

\aux{@hogAño}{}{\ent}{ord(HOGAÑO)}

\aux{@hogTrimestre}{}{\ent}{ord(HOGTRIMESTRE)}

\aux{@hogLatitud}{}{\ent}{ord(HOGLATITUD)}

\aux{@hogLongitud}{}{\ent}{ord(HOGLONGITUD)}

\aux{@ii7}{}{\ent}{ord(II7)}

\aux{@region}{}{\ent}{ord(REGION)}

\aux{@mas500}{}{\ent}{ord(MAS 500)}

\aux{@iv1}{}{\ent}{ord(IV1)}

\aux{@iv2}{}{\ent}{ord(IV2)}

\aux{@ii2}{}{\ent}{ord(II2)}

\aux{@ii3}{}{\ent}{ord(II3)}

\comentario{Tabla Personas}

\aux{@indCodusu}{}{\ent}{ord(INDCODUSU)}

\aux{@componente}{}{\ent}{ord(COMPONENTE)}

\aux{@indAño}{}{\ent}{ord(INDAÑO)}

\aux{@indTrimestre}{}{\ent}{ord(INDTRIMESTRE)}

\aux{@ch4}{}{\ent}{ord(CH4)}

\aux{@ch6}{}{\ent}{ord(CH6)}

\aux{@nivelEd}{}{\ent}{ord(NIVEL ED)}

\aux{@estado}{}{\ent}{ord(ESTADO)}

\aux{@catOcup}{}{\ent}{ord(CAT OCUP)}

\aux{@p47t}{}{\ent}{ord(p47T)}

\aux{@pp04g}{}{\ent}{ord(PP04G)}

\section{Especificación}

\begin{align*}
	 & esHogarMatriz(th) \wedge esIndividuoMatriz(ti) \wedge                             \\
	 & \longitud{th} > 0 \wedge \longitud{ti} > 0 \wedge                                 \\
	 & esTablaHogarCompleta(th) \wedge esTablaIndividuoCompleta(ti) \wedge               \\
	 & \neg hayIndividuoSinHogar(th, ti) \wedge \neg hayHogarSinIndividuo(th, ti) \wedge \\
	 & \neg hayIndividuoRepetido(ti) \wedge \neg hayHogarRepetido(th) \wedge             \\
	 & localizacionesValidas(th) \wedge                                                  \\
	 & mismaFechaDeRelevamiento(th, ti) \wedge                                           \\
	 & todosHogaresConMenosDe21Integrantes(th, ti) \wedge                                \\
	 & todosHogaresConMasHabitacionesQueDormitorios(th, ti) \wedge                       \\
	 & todosAtributosEnRango(th, ti)                                                     \\
\end{align*}

\begin{proc}{esEncuestaVálida}{\In th: $eph_h$, \In ti: $eph_i$, \Out result: Bool}{}
	\pre{ True }
	\post{result = true \Iff}
\end{proc}

\pred{esHogarMatriz}{s: $eph_h$}{
	\paraTodo{i}{\ent}{
		0\leq i < filas(s) \implicaLuego (\longitud{s[i]} > 0 \wedge \newline \paraTodo{j}{\ent}{
			0\leq j < filas(s) \implicaLuego \longitud{m[i]} = \longitud{m[j]}
		}{})
	}{}
}

\pred{esIndividuoMatriz}{s: $eph_i$}{
	\paraTodo{i}{\ent}{
		0\leq i < filas(s) \implicaLuego (\longitud{s[i]} > 0 \wedge \newline \paraTodo{j}{\ent}{
			0\leq j < filas(s) \implicaLuego \longitud{m[i]} = \longitud{m[j]}
		}{})
	}{}
}

\pred{hayIndividuoRepetido}{s: $eph_i$}{
	\existe{a,b}{individuo}{(a\in s \wedge b\in s) \wedge (a[@indCodusu] = b[@indCodusu] \wedge a[@componente] = b[@componente])}{}
}

\pred{hayHogarRepetido}{s: $eph_h$}{
	\existe{a,b}{hogar}{(a\in s \wedge b\in s) \wedge (a[@hogCodusu] = b[@hogCodusu])}{}
}

\end{document}