\documentclass[11pt]{article}

\usepackage{ifthen}
\usepackage{amssymb}
\usepackage{multicol}
\usepackage{graphicx}
\usepackage[absolute]{textpos}
\usepackage{amsmath, amscd, amssymb, amsthm, latexsym}
% \usepackage[noload]{qtree}
%\usepackage{xspace,rotating,calligra,dsfont,ifthen}
\usepackage{xspace,rotating,dsfont,ifthen}
\usepackage[spanish,activeacute]{babel}
\usepackage[utf8]{inputenc}
\usepackage{pgfpages}
\usepackage{pgf,pgfarrows,pgfnodes,pgfautomata,pgfheaps,xspace,dsfont}
\usepackage{listings}
\usepackage{multicol}
\usepackage{todonotes}
\usepackage{url}
\usepackage{float}
\usepackage{framed,mdframed}
\usepackage{cancel}

\usepackage[strict]{changepage}


\makeatletter


\newcommand\hfrac[2]{\genfrac{}{}{0pt}{}{#1}{#2}} %\hfrac{}{} es un \frac sin la linea del medio

\newcommand\Wider[2][3em]{% \Wider[3em]{} reduce los m\'argenes
\makebox[\linewidth][c]{%
  \begin{minipage}{\dimexpr\textwidth+#1\relax}
  \raggedright#2
  \end{minipage}%
  }%
}


\@ifclassloaded{beamer}{%
  \newcommand{\tocarEspacios}{%
    \addtolength{\leftskip}{4em}%
    \addtolength{\parindent}{-3em}%
  }%
}
{%
  \usepackage[top=1cm,bottom=2cm,left=1cm,right=1cm]{geometry}%
  \usepackage{color}%
  \newcommand{\tocarEspacios}{%
    \addtolength{\leftskip}{3em}%
    \setlength{\parindent}{0em}%
  }%
}

\newcommand{\encabezadoDeProc}[4]{%
  % Ponemos la palabrita problema en tt
%  \noindent%
  {\normalfont\bfseries\ttfamily proc}%
  % Ponemos el nombre del problema
  \ %
  {\normalfont\ttfamily #2}%
  \
  % Ponemos los parametros
  (#3)%
  \ifthenelse{\equal{#4}{}}{}{%
  \ =\ %
  % Ponemos el nombre del resultado
  {\normalfont\ttfamily #1}%
  % Por ultimo, va el tipo del resultado
  \ : #4}
}

\newcommand{\encabezadoDeTipo}[2]{%
  % Ponemos la palabrita tipo en tt
  {\normalfont\bfseries\ttfamily tipo}%
  % Ponemos el nombre del tipo
  \ %
  {\normalfont\ttfamily #2}%
  \ifthenelse{\equal{#1}{}}{}{$\langle$#1$\rangle$}
}

% Primero definiciones de cosas al estilo title, author, date

\def\materia#1{\gdef\@materia{#1}}
\def\@materia{No especifi\'o la materia}
\def\lamateria{\@materia}

\def\cuatrimestre#1{\gdef\@cuatrimestre{#1}}
\def\@cuatrimestre{No especifi\'o el cuatrimestre}
\def\elcuatrimestre{\@cuatrimestre}

\def\anio#1{\gdef\@anio{#1}}
\def\@anio{No especifi\'o el anio}
\def\elanio{\@anio}

\def\fecha#1{\gdef\@fecha{#1}}
\def\@fecha{\today}
\def\lafecha{\@fecha}

\def\nombre#1{\gdef\@nombre{#1}}
\def\@nombre{No especific'o el nombre}
\def\elnombre{\@nombre}

\def\practicas#1{\gdef\@practica{#1}}
\def\@practica{No especifi\'o el n\'umero de pr\'actica}
\def\lapractica{\@practica}


% Esta macro convierte el numero de cuatrimestre a palabras
\newcommand{\cuatrimestreLindo}{
  \ifthenelse{\equal{\elcuatrimestre}{1}}
  {Primer cuatrimestre}
  {\ifthenelse{\equal{\elcuatrimestre}{2}}
  {Segundo cuatrimestre}
  {Verano}}
}


\newcommand{\depto}{{UBA -- Facultad de Ciencias Exactas y Naturales --
      Departamento de Computaci\'on}}

\newcommand{\titulopractica}{
  \centerline{\depto}
  \vspace{1ex}
  \centerline{{\Large\lamateria}}
  \vspace{0.5ex}
  \centerline{\cuatrimestreLindo de \elanio}
  \vspace{2ex}
  \centerline{{\huge Pr\'actica \lapractica -- \elnombre}}
  \vspace{5ex}
  \arreglarincisos
  \newcounter{ejercicio}
  \newenvironment{ejercicio}{\stepcounter{ejercicio}\textbf{Ejercicio
      \theejercicio}%
    \renewcommand\@currentlabel{\theejercicio}%
  }{\vspace{0.2cm}}
}


\newcommand{\titulotp}{
  \centerline{\depto}
  \vspace{1ex}
  \centerline{{\Large\lamateria}}
  \vspace{0.5ex}
  \centerline{\cuatrimestreLindo de \elanio}
  \vspace{0.5ex}
  \centerline{\lafecha}
  \vspace{2ex}
  \centerline{{\huge\elnombre}}
  \vspace{5ex}
}


%practicas
\newcommand{\practica}[2]{%
    \title{Pr\'actica #1 \\ #2}
    \author{Algoritmos y Estructuras de Datos I}
    \date{Segundo Cuatrimestre 2019}

    \maketitlepractica{#1}{#2}
}

\newcommand \maketitlepractica[2] {%
\begin{center}
\begin{tabular}{r cr}
 \begin{tabular}{c}
{\large\bf\textsf{\ Algoritmos y Estructuras de Datos I\ }}\\
Primer Cuatrimestre 2021\\
\title{\normalsize Gu\'ia Pr\'actica #1 \\ \textbf{#2}}\\
\@title
\end{tabular} &
\begin{tabular}{@{} p{1.6cm} @{}}
\includegraphics[width=1.6cm]{logodpt.jpg}
\end{tabular} &
\begin{tabular}{l @{}}
 \emph{Departamento de Computaci\'on} \\
 \emph{Facultad de Ciencias Exactas y Naturales} \\
 \emph{Universidad de Buenos Aires} \\
\end{tabular}
\end{tabular}
\end{center}

\bigskip
}


% Símbolos varios

\newcommand{\nat}{\ensuremath{\mathds{N}}}
\newcommand{\ent}{\ensuremath{\mathds{Z}}}
\newcommand{\float}{\ensuremath{\mathds{R}}}
\newcommand{\bool}{\ensuremath{\mathsf{Bool}}}
\newcommand{\True}{\ensuremath{\mathrm{true}}}
\newcommand{\False}{\ensuremath{\mathrm{false}}}
\newcommand{\Then}{\ensuremath{\rightarrow}}
\newcommand{\Iff}{\ensuremath{\leftrightarrow}}
\newcommand{\implica}{\ensuremath{\longrightarrow}}
\newcommand{\IfThenElse}[3]{\ensuremath{\mathsf{if}\ #1\ \mathsf{then}\ #2\ \mathsf{else}\ #3\ \mathsf{fi}}}
\newcommand{\In}{\textsf{in }}
\newcommand{\Out}{\textsf{out }}
\newcommand{\Inout}{\textsf{inout }}
\newcommand{\yLuego}{\land _L}
\newcommand{\oLuego}{\lor _L}
\newcommand{\implicaLuego}{\implica _L}
\newcommand{\cuantificador}[5]{%
	\ensuremath{(#2 #3: #4)\ (%
		\ifthenelse{\equal{#1}{unalinea}}{
			#5
		}{
			$ % exiting math mode
			\begin{adjustwidth}{+2em}{}
			$#5$%
			\end{adjustwidth}%
			$ % entering math mode
		}
	)}
}

\newcommand{\existe}[4][]{%
	\cuantificador{#1}{\exists}{#2}{#3}{#4}
}
\newcommand{\paraTodo}[4][]{%
	\cuantificador{#1}{\forall}{#2}{#3}{#4}
}

% Símbolo para marcar los ejercicios importantes (estrellita)
\newcommand\importante{\raisebox{0.5pt}{\ensuremath{\bigstar}}}


\newcommand{\rango}[2]{[#1\twodots#2]}
\newcommand{\comp}[2]{[\,#1\,|\,#2\,]}

\newcommand{\rangoac}[2]{(#1\twodots#2]}
\newcommand{\rangoca}[2]{[#1\twodots#2)}
\newcommand{\rangoaa}[2]{(#1\twodots#2)}

%ejercicios
\newtheorem{exercise}{Ejercicio}
\newenvironment{ejercicio}[1][]{\begin{exercise}#1\rm}{\end{exercise} \vspace{0.2cm}}
\newenvironment{items}{\begin{enumerate}[a)]}{\end{enumerate}}
\newenvironment{subitems}{\begin{enumerate}[i)]}{\end{enumerate}}
\newcommand{\sugerencia}[1]{\noindent \textbf{Sugerencia:} #1}

\lstnewenvironment{code}{
    \lstset{% general command to set parameter(s)
        language=C++, basicstyle=\small\ttfamily, keywordstyle=\slshape,
        emph=[1]{tipo,usa}, emphstyle={[1]\sffamily\bfseries},
        morekeywords={tint,forn,forsn},
        basewidth={0.47em,0.40em},
        columns=fixed, fontadjust, resetmargins, xrightmargin=5pt, xleftmargin=15pt,
        flexiblecolumns=false, tabsize=2, breaklines, breakatwhitespace=false, extendedchars=true,
        numbers=left, numberstyle=\tiny, stepnumber=1, numbersep=9pt,
        frame=l, framesep=3pt,
    }
   \csname lst@SetFirstLabel\endcsname}
  {\csname lst@SaveFirstLabel\endcsname}


%tipos basicos
\newcommand{\rea}{\ensuremath{\mathsf{Float}}}
\newcommand{\cha}{\ensuremath{\mathsf{Char}}}
\newcommand{\str}{\ensuremath{\mathsf{String}}}

\newcommand{\mcd}{\mathrm{mcd}}
\newcommand{\prm}[1]{\ensuremath{\mathsf{prm}(#1)}}
\newcommand{\sgd}[1]{\ensuremath{\mathsf{sgd}(#1)}}

\newcommand{\tuple}[2]{\ensuremath{#1 \times #2}}

%listas
\newcommand{\TLista}[1]{\ensuremath{seq \langle #1\rangle}}
\newcommand{\lvacia}{\ensuremath{[\ ]}}
\newcommand{\lv}{\ensuremath{[\ ]}}
\newcommand{\longitud}[1]{\ensuremath{|#1|}}
\newcommand{\cons}[1]{\ensuremath{\mathsf{addFirst}}(#1)}
\newcommand{\indice}[1]{\ensuremath{\mathsf{indice}}(#1)}
\newcommand{\conc}[1]{\ensuremath{\mathsf{concat}}(#1)}
\newcommand{\cab}[1]{\ensuremath{\mathsf{head}}(#1)}
\newcommand{\cola}[1]{\ensuremath{\mathsf{tail}}(#1)}
\newcommand{\sub}[1]{\ensuremath{\mathsf{subseq}}(#1)}
\newcommand{\en}[1]{\ensuremath{\mathsf{en}}(#1)}
\newcommand{\cuenta}[2]{\mathsf{cuenta}\ensuremath{(#1, #2)}}
\newcommand{\suma}[1]{\mathsf{suma}(#1)}
\newcommand{\twodots}{\ensuremath{\mathrm{..}}}
\newcommand{\masmas}{\ensuremath{++}}
\newcommand{\matriz}[1]{\TLista{\TLista{#1}}}

\newcommand{\seqchar}{\TLista{\cha}}


% Acumulador
\newcommand{\acum}[1]{\ensuremath{\mathsf{acum}}(#1)}
\newcommand{\acumselec}[3]{\ensuremath{\mathrm{acum}(#1 |  #2, #3)}}

% \selector{variable}{dominio}
\newcommand{\selector}[2]{#1~\ensuremath{\leftarrow}~#2}
\newcommand{\selec}{\ensuremath{\leftarrow}}

\newcommand{\pred}[3]{%
    {\normalfont\bfseries\ttfamily\noindent pred }%
    {\normalfont\ttfamily #1}%
    \ifthenelse{\equal{#2}{}}{}{\ (#2) }%
    \{%
    \begin{adjustwidth}{+2em}{}
      \ensuremath{#3}
    \end{adjustwidth}
    \}%
    {\normalfont\bfseries\,\par}%
}

\newenvironment{proc}[4][res]{%

  % El parametro 1 (opcional) es el nombre del resultado
  % El parametro 2 es el nombre del problema
  % El parametro 3 son los parametros
  % El parametro 4 es el tipo del resultado
  % Preambulo del ambiente problema
  % Tenemos que definir los comandos requiere, asegura, modifica y aux
  \newcommand{\pre}[2][]{%
    {\normalfont\bfseries\ttfamily Pre}%
    \ifthenelse{\equal{##1}{}}{}{\ {\normalfont\ttfamily ##1} :}\ %
    \{\ensuremath{##2}\}%
    {\normalfont\bfseries\,\par}%
  }
  \newcommand{\post}[2][]{%
    {\normalfont\bfseries\ttfamily Post}%
    \ifthenelse{\equal{##1}{}}{}{\ {\normalfont\ttfamily ##1} :}\
    \{\ensuremath{##2}\}%
    {\normalfont\bfseries\,\par}%
  }
  \renewcommand{\aux}[4]{%
    {\normalfont\bfseries\ttfamily aux\ }%
    {\normalfont\ttfamily ##1}%
    \ifthenelse{\equal{##2}{}}{}{\ (##2)}\ : ##3\, = \ensuremath{##4}%
    {\normalfont\bfseries\,;\par}%
  }
  \renewcommand{\pred}[3]{%
    {\normalfont\bfseries\ttfamily pred }%
    {\normalfont\ttfamily ##1}%
    \ifthenelse{\equal{##2}{}}{}{\ (##2) }%
    \{%
    \begin{adjustwidth}{+5em}{}
      \ensuremath{##3}
    \end{adjustwidth}
    \}%
    {\normalfont\bfseries\,\par}%
  }

  \newcommand{\res}{#1}
  \vspace{1ex}
  \noindent
  \encabezadoDeProc{#1}{#2}{#3}{#4}
  % Abrimos la llave
  \{\par%
  \tocarEspacios
}
% Ahora viene el cierre del ambiente problema
{
  % Cerramos la llave
  \noindent\}
  \vspace{1ex}
}


\newcommand{\aux}[4]{%
    {\normalfont\bfseries\ttfamily\noindent aux\ }%
    {\normalfont\ttfamily #1}%
    \ifthenelse{\equal{#2}{}}{}{\ (#2)}\ : #3\, = \ensuremath{#4}%
    {\normalfont\bfseries\,;\par}%
}


% \newcommand{\pre}[1]{\textsf{pre}\ensuremath{(#1)}}

\newcommand{\procnom}[1]{\textsf{#1}}
\newcommand{\procil}[3]{\textsf{proc #1}\ensuremath{(#2) = #3}}
\newcommand{\procilsinres}[2]{\textsf{proc #1}\ensuremath{(#2)}}
\newcommand{\preil}[2]{\textsf{Pre #1: }\ensuremath{#2}}
\newcommand{\postil}[2]{\textsf{Post #1: }\ensuremath{#2}}
\newcommand{\auxil}[2]{\textsf{fun }\ensuremath{#1 = #2}}
\newcommand{\auxilc}[4]{\textsf{fun }\ensuremath{#1( #2 ): #3 = #4}}
\newcommand{\auxnom}[1]{\textsf{fun }\ensuremath{#1}}
\newcommand{\auxpred}[3]{\textsf{pred }\ensuremath{#1( #2 ) \{ #3 \}}}

\newcommand{\comentario}[1]{{/*\ #1\ */}}

\newcommand{\nom}[1]{\ensuremath{\mathsf{#1}}}


% En las practicas/parciales usamos numeros arabigos para los ejercicios.
% Aca cambiamos los enumerate comunes para que usen letras y numeros
% romanos
\newcommand{\arreglarincisos}{%
  \renewcommand{\theenumi}{\alph{enumi}}
  \renewcommand{\theenumii}{\roman{enumii}}
  \renewcommand{\labelenumi}{\theenumi)}
  \renewcommand{\labelenumii}{\theenumii)}
}



%%%%%%%%%%%%%%%%%%%%%%%%%%%%%% PARCIAL %%%%%%%%%%%%%%%%%%%%%%%%
\let\@xa\expandafter
\newcommand{\tituloparcial}{\centerline{\depto -- \lamateria}
  \centerline{\elnombre -- \lafecha}%
  \setlength{\TPHorizModule}{10mm} % Fija las unidades de textpos
  \setlength{\TPVertModule}{\TPHorizModule} % Fija las unidades de
                                % textpos
  \arreglarincisos
  \newcounter{total}% Este contador va a guardar cuantos incisos hay
                    % en el parcial. Si un ejercicio no tiene incisos,
                    % cuenta como un inciso.
  \newcounter{contgrilla} % Para hacer ciclos
  \newcounter{columnainicial} % Se van a usar para los cline cuando un
  \newcounter{columnafinal}   % ejercicio tenga incisos.
  \newcommand{\primerafila}{}
  \newcommand{\segundafila}{}
  \newcommand{\rayitas}{} % Esto va a guardar los \cline de los
                          % ejercicios con incisos, asi queda mas bonito
  \newcommand{\anchodegrilla}{20} % Es para textpos
  \newcommand{\izquierda}{7} % Estos dos le dicen a textpos donde colocar
  \newcommand{\abajo}{2}     % la grilla
  \newcommand{\anchodecasilla}{0.4cm}
  \setcounter{columnainicial}{1}
  \setcounter{total}{0}
  \newcounter{ejercicio}
  \setcounter{ejercicio}{0}
  \renewenvironment{ejercicio}[1]
  {%
    \stepcounter{ejercicio}\textbf{\noindent Ejercicio \theejercicio. [##1
      puntos]}% Formato
    \renewcommand\@currentlabel{\theejercicio}% Esto es para las
                                % referencias
    \newcommand{\invariante}[2]{%
      {\normalfont\bfseries\ttfamily invariante}%
      \ ####1\hspace{1em}####2%
    }%
    \newcommand{\Proc}[5][result]{
      \encabezadoDeProc{####1}{####2}{####3}{####4}\hspace{1em}####5}%
  }% Aca se termina el principio del ejercicio
  {% Ahora viene el final
    % Esto suma la cantidad de incisos o 1 si no hubo ninguno
    \ifthenelse{\equal{\value{enumi}}{0}}
    {\addtocounter{total}{1}}
    {\addtocounter{total}{\value{enumi}}}
    \ifthenelse{\equal{\value{ejercicio}}{1}}{}
    {
      \g@addto@macro\primerafila{&} % Si no estoy en el primer ej.
      \g@addto@macro\segundafila{&}
    }
    \ifthenelse{\equal{\value{enumi}}{0}}
    {% No tiene incisos
      \g@addto@macro\primerafila{\multicolumn{1}{|c|}}
      \bgroup% avoid overwriting somebody else's value of \tmp@a
      \protected@edef\tmp@a{\theejercicio}% expand as far as we can
      \@xa\g@addto@macro\@xa\primerafila\@xa{\tmp@a}%
      \egroup% restore old value of \tmp@a, effect of \g@addto.. is

      \stepcounter{columnainicial}
    }
    {% Tiene incisos
      % Primero ponemos el encabezado
      \g@addto@macro\primerafila{\multicolumn}% Ahora el numero de items
      \bgroup% avoid overwriting somebody else's value of \tmp@a
      \protected@edef\tmp@a{\arabic{enumi}}% expand as far as we can
      \@xa\g@addto@macro\@xa\primerafila\@xa{\tmp@a}%
      \egroup% restore old value of \tmp@a, effect of \g@addto.. is
      % global
      % Ahora el formato
      \g@addto@macro\primerafila{{|c|}}%
      % Ahora el numero de ejercicio
      \bgroup% avoid overwriting somebody else's value of \tmp@a
      \protected@edef\tmp@a{\theejercicio}% expand as far as we can
      \@xa\g@addto@macro\@xa\primerafila\@xa{\tmp@a}%
      \egroup% restore old value of \tmp@a, effect of \g@addto.. is
      % global
      % Ahora armamos la segunda fila
      \g@addto@macro\segundafila{\multicolumn{1}{|c|}{a}}%
      \setcounter{contgrilla}{1}
      \whiledo{\value{contgrilla}<\value{enumi}}
      {%
        \stepcounter{contgrilla}
        \g@addto@macro\segundafila{&\multicolumn{1}{|c|}}
        \bgroup% avoid overwriting somebody else's value of \tmp@a
        \protected@edef\tmp@a{\alph{contgrilla}}% expand as far as we can
        \@xa\g@addto@macro\@xa\segundafila\@xa{\tmp@a}%
        \egroup% restore old value of \tmp@a, effect of \g@addto.. is
        % global
      }
      % Ahora armo las rayitas
      \setcounter{columnafinal}{\value{columnainicial}}
      \addtocounter{columnafinal}{-1}
      \addtocounter{columnafinal}{\value{enumi}}
      \bgroup% avoid overwriting somebody else's value of \tmp@a
      \protected@edef\tmp@a{\noexpand\cline{%
          \thecolumnainicial-\thecolumnafinal}}%
      \@xa\g@addto@macro\@xa\rayitas\@xa{\tmp@a}%
      \egroup% restore old value of \tmp@a, effect of \g@addto.. is
      \setcounter{columnainicial}{\value{columnafinal}}
      \stepcounter{columnainicial}
    }
    \setcounter{enumi}{0}%
    \vspace{0.2cm}%
  }%
  \newcommand{\tercerafila}{}
  \newcommand{\armartercerafila}{
    \setcounter{contgrilla}{1}
    \whiledo{\value{contgrilla}<\value{total}}
    {\stepcounter{contgrilla}\g@addto@macro\tercerafila{&}}
  }
  \newcommand{\grilla}{%
    \g@addto@macro\primerafila{&\textbf{TOTAL}}
    \g@addto@macro\segundafila{&}
    \g@addto@macro\tercerafila{&}
    \armartercerafila
    \ifthenelse{\equal{\value{total}}{\value{ejercicio}}}
    {% No hubo incisos
      \begin{textblock}{\anchodegrilla}(\izquierda,\abajo)
        \begin{tabular}{|*{\value{total}}{p{\anchodecasilla}|}c|}
          \hline
          \primerafila\\
          \hline
          \tercerafila\\
          \tercerafila\\
          \hline
        \end{tabular}
      \end{textblock}
    }
    {% Hubo incisos
      \begin{textblock}{\anchodegrilla}(\izquierda,\abajo)
        \begin{tabular}{|*{\value{total}}{p{\anchodecasilla}|}c|}
          \hline
          \primerafila\\
          \rayitas
          \segundafila\\
          \hline
          \tercerafila\\
          \tercerafila\\
          \hline
        \end{tabular}
      \end{textblock}
    }
  }%
  % \datosalumno
}

\newcommand{\datosalumno}{
  \vspace{0.4cm}
  \textbf{Apellidos:}

  \textbf{Nombres:}

  \textbf{LU:}

  \textbf{Correo electrónico:}

  \textbf{Nro. de carillas que adjunta:}
  \vspace{0.5cm}
}


% AMBIENTE CONSIGNAS
% Se usa en el TP para ir agregando las cosas que tienen que resolver
% los alumnos.
% Dentro del ambiente hay que usar \item para cada consigna

\newcounter{consigna}
\setcounter{consigna}{0}

\newenvironment{consignas}{%
  \newcommand{\consigna}{\stepcounter{consigna}\textbf{\theconsigna.}}%
  \renewcommand{\ejercicio}[1]{\item ##1 }
  \renewcommand{\proc}[5][result]{\item
    \encabezadoDeProc{##1}{##2}{##3}{##4}\hspace{1em}##5}%
  \newcommand{\invariante}[2]{\item%
    {\normalfont\bfseries\ttfamily invariante}%
    \ ##1\hspace{1em}##2%
  }
  \renewcommand{\aux}[4]{\item%
    {\normalfont\bfseries\ttfamily aux\ }%
    {\normalfont\ttfamily ##1}%
    \ifthenelse{\equal{##2}{}}{}{\ (##2)}\ : ##3 \hspace{1em}##4%
  }
  % Comienza la lista de consignas
  \begin{list}{\consigna}{%
      \setlength{\itemsep}{0.5em}%
      \setlength{\parsep}{0cm}%
    }
}%
{\end{list}}



% para decidir si usar && o ^
\newcommand{\y}[0]{\ensuremath{\land}}

% macros de correctitud
\newcommand{\semanticComment}[2]{#1 \ensuremath{#2};}
\newcommand{\namedSemanticComment}[3]{#1 #2: \ensuremath{#3};}


\newcommand{\local}[1]{\semanticComment{local}{#1}}

\newcommand{\vale}[1]{\semanticComment{vale}{#1}}
\newcommand{\valeN}[2]{\namedSemanticComment{vale}{#1}{#2}}
\newcommand{\impl}[1]{\semanticComment{implica}{#1}}
\newcommand{\implN}[2]{\namedSemanticComment{implica}{#1}{#2}}
\newcommand{\estado}[1]{\semanticComment{estado}{#1}}

\newcommand{\invarianteCN}[2]{\namedSemanticComment{invariante}{#1}{#2}}
\newcommand{\invarianteC}[1]{\semanticComment{invariante}{#1}}
\newcommand{\varianteCN}[2]{\namedSemanticComment{variante}{#1}{#2}}
\newcommand{\varianteC}[1]{\semanticComment{variante}{#1}}

\usepackage{caratula}
\usepackage[hidelinks]{hyperref}

\newcommand{\br}{\hfill\break}
\newcommand{\bloque}[2]{
	\ensuremath{
		$\begin{adjustwidth}{#1}{}
			$#2$
		\end{adjustwidth}
		$
	}
}

\begin{document}
%Caratula
\titulo{Trabajo Práctico de Especificación}
\subtitulo{Análisis Habitacional Argentino}
\fecha{08/09/2021}
\materia{Lc. Computación}
\integrante{Simon, Martino}{374/21}{martinosimon@gmail.com}
\integrante{Ramis, Ezequiel}{881/21}{ezequielramis.hello@gmail.com}
\maketitle
%Creación de indice
\tableofcontents
\newpage

\section{Auxiliares para base de datos y generales}

Antes de empezar con la especificación de procedimientos, vamos a definir
auxiliares para acceder más declarativamente a los atributos de cada tabla, y
para uso general: \\ \br\comentario{Tabla Hogares}\\

\aux{@hogCodusu}{}{\ent}{ord(HOGCODUSU)}

\aux{@hogAño}{}{\ent}{ord(HOGA\tilde NO)}

\aux{@hogTrimestre}{}{\ent}{ord(HOGTRIMESTRE)}

\aux{@hogLatitud}{}{\ent}{ord(HOGLATITUD)}

\aux{@hogLongitud}{}{\ent}{ord(HOGLONGITUD)}

\aux{@ii7}{}{\ent}{ord(II7)}

\aux{@region}{}{\ent}{ord(REGION)}

\aux{@mas500}{}{\ent}{ord(MAS\_500)}

\aux{@iv1}{}{\ent}{ord(IV1)}

\aux{@iv2}{}{\ent}{ord(IV2)}

\aux{@ii2}{}{\ent}{ord(II2)}

\aux{@ii3}{}{\ent}{ord(II3)}

\br\comentario{Tabla Personas}\\

\aux{@indCodusu}{}{\ent}{ord(INDCODUSU)}

\aux{@componente}{}{\ent}{ord(COMPONENTE)}

\aux{@indAño}{}{\ent}{ord(INDA\tilde NO)}

\aux{@indTrimestre}{}{\ent}{ord(INDTRIMESTRE)}

\aux{@ch4}{}{\ent}{ord(CH4)}

\aux{@ch6}{}{\ent}{ord(CH6)}

\aux{@nivelEd}{}{\ent}{ord(NIVEL ED)}

\aux{@estado}{}{\ent}{ord(ESTADO)}

\aux{@catOcup}{}{\ent}{ord(CAT OCUP)}

\aux{@p47t}{}{\ent}{ord(P47T)}

\aux{@pp04g}{}{\ent}{ord(PP04G)}

\br\comentario{Varios}\\

\aux{\#atributosHogar}{}{\ent}{12}

\aux{\#atributosPersona}{}{\ent}{11}

\aux{\#habitaciones}{h: hogar}{dato}{h[@iv2]}

\aux{\#dormitorios}{h: hogar}{dato}{h[@ii2]}

\aux{\#personasEnHogar}{ti: $eph_i$, h: hogar}{\ent}{
	\sum_{i\in ti} \IfThenElse{personaEnHogar(i, h)}{1}{0}
}

\pred{esCasa}{h: hogar}{
	h[@iv1]=1
}

\pred{personaEnHogar}{i: individuo, h: hogar} {
	i[@indCodusu] = h[@hogCodusu]
}

\pred{hogaresConTrimestresIguales}{h: hogar, g: hogar}{
	h[@hogTrimestre] = g[@hogTrimestre]
}

\section{Especificación}
\subsection{Ejercicio 1}

\begin{proc}{esEncuestaVálida}{\In th: $eph_h$, \In ti: $eph_i$, \Out result: \bool}{}
	\pre{ True }
	\post{ result=true \Iff encuestaValida(th, ti) }
\end{proc}

\pred{encuestaValida}{th: $eph_h$, ti: $eph_i$}{
	% Listo
	esMatriz(th) \wedge esMatriz(ti) \wedge\\
	% Listo
	|th| > 0 \wedge |ti| > 0 \wedge\\
	% Listo
	esTablaCompleta(th, \#atributosHogar) \wedge esTablaCompleta(ti, \#atributosPersona) \wedge\\
	% Listo
	\neg hayIndividuoSinHogar(th, ti) \wedge \neg hayHogarSinIndividuo(th, ti) \wedge\\
	% Listo
	\neg hayIndividuoRepetido(ti) \wedge \neg hayHogarRepetido(th) \wedge\\
	% Listo
	mismaFechaDeRelevamiento(th, ti) \wedge\\
	% Listo
	todosHogaresConMenosDe21Integrantes(th, ti) \wedge\\
	% Listo
	todosHogaresConMasHabitacionesQueDormitorios(th) \wedge\\
	% Listo
	todosAtributosEnRango(th, ti) \wedge\\
	atributosValidos(th, ti)
}

\pred{esMatriz}{s: \matriz{dato}}{
	|s| > 0 \implicaLuego
	\neg\existe{t}{\TLista{dato}}{
		(t\in s) \wedge |head(s)|\neq |t|
	}
}

\pred{esTablaCompleta}{s: \matriz{dato}, l: \ent}{
	\paraTodo{f}{\TLista{dato}}{
		f\in s \implica |f| = l
	}
}

\pred{hayIndividuoSinHogar}{th: $eph_h$, ti: $eph_i$}{
	\existe{i}{individuo}{
		i\in ti \wedge \bloque{+1em}{
			\neg\existe{h}{hogar}{
				h\in th \wedge personaEnHogar(i, h)
			}
		}
	}
}

\pred{hayHogarSinIndividuo}{th: $eph_h$, ti: $eph_i$}{
	\existe{h}{hogar}{
		h\in th \wedge \bloque{+1em}{
			\neg\existe{i}{individuo}{
				i\in ti \wedge personaEnHogar(i, h)
			}
		}
	}
}

\pred{hayIndividuoRepetido}{s: $eph_i$}{
	\existe{a,b}{individuo}{(a\in s \wedge b\in s) \wedge
		a[@indCodusu] = b[@indCodusu] \wedge
		a[@componente] = b[@componente]}
}

\pred{hayHogarRepetido}{s: $eph_h$}{
	\existe{a,b}{hogar}{(a\in s \wedge b\in s) \wedge
		a[@hogCodusu] = b[@hogCodusu]}
}

\pred{mismaFechaDeRelevamiento}{th: $eph_h$, ti: $eph_i$}{
	(|th| > 0 \wedge |ti| > 0) \implicaLuego
	(\bloque{+2em}{
		\paraTodo{h}{hogar}{
			h\in th \implica (\bloque{+2em}{
				h[@hogA\tilde no] = head(h)[@hogA\tilde no] \wedge\\
				hogaresConTrimestresIguales(h, head(th))
			})
		} \wedge\\
		\paraTodo{i}{individuo}{
			i\in ti \implica (\bloque{+2em}{
				i[@indA\tilde n o] = head(th)[@hogA\tilde no] \wedge\\
				i[@indTrimestre] = head(th)[@hogTrimestre]
			})
		}
	})
}

\pred{todosHogaresConMenosDe21Integrantes}{th: $eph_h$, ti:
	$eph_i$}{
	\paraTodo{h}{hogar}{
		h\in th \implica 21 > \#personasEnHogar(ti, h)
	}
}

\pred{todosHogaresConMasHabitacionesQueDormitorios}{th: $eph_h$}{
	\paraTodo{h}{hogar}{h\in th \implica \#habitaciones(h) \geq \#dormitorios(h)}
}

\pred{todosAtributosEnRango}{th: $eph_h$, ti:
	$eph_i$}{
	\paraTodo{h}{hogar}{
		1\leq h[@ii7]\leq 3 \wedge\\ 1\leq
		h[@region]\leq 6 \wedge\\ 0\leq
		h[@mas500]\leq 1\wedge\\ 1\leq
		h[@iv1]\leq 5\wedge\\ 1\leq
		h[@ii3]\leq 2
	}\wedge\\
	\paraTodo {i}{individuo}{
		1\leq i[@ch4]\leq 2 \wedge\\ 0\leq
		i[@nivelEd]\leq 1 \wedge\\ -1\leq
		i[@estado]\leq 1 \wedge\\ 0\leq
		i[@catOcup]\leq 4 \wedge\\ 1\leq
		i[@pp04g]\leq 10 \wedge \\ -1\leq
		i[@p47t]
	}
}

\pred{atributosValidos}{th: $eph_h$, ti: $eph_i$}{
	\paraTodo{h}{hogar}{
		1800\leq h[@hogA \tilde n o] \wedge \\ 1\leq
		h[@hogTrimestre]\leq 4 \wedge \\ 1\leq \#habitaciones(h) \wedge
		\\ 1\leq \#dormitorios(h)
	}\wedge\\
	\paraTodo {i}{individuo}{
		1\leq i[@componente] \wedge \\ 1800\leq
		i[@indA\tilde n o] \wedge \\ 1\leq
		i[@indTrimestre]\leq 4 \wedge \\ 0\leq
		i[@ch6]
	}
}

\subsection{Ejercicio 2}

\begin{proc}{histHabitacional}{\In th: $eph_h$, \In ti: $eph_i$, \In region: \ent, \Out res: \TLista{\ent}}{}
	\pre{ encuestaValida(th, ti) \wedge 1\leq region\leq 6 \wedge (\exists{h: hogar})(h\in th \wedge h[@region] = region \wedge esCasa(h))}
	\post{ longitudDeHistograma(th, res) \yLuego
		\bloque{+6em}{
			(\forall{i: \ent})
			(0\leq i < |res| \implicaLuego res[i] =	\#casaConNHabitacionesEnRegion(th, i + 1, region))
		}
	}
\end{proc}

\aux{\#casaConNHabitacionesEnRegion}{th: $eph_h$, n: \ent, r: dato}{\ent}{
	\bloque{+2em}{
		\sum_{h\in th}\IfThenElse{esCasa(h) \wedge \#habitaciones(h) = n \wedge h[@region]=r}{1}{0}
	}
}

\pred{longitudDeHistograma}{th: $eph_h$, res: \TLista{\ent}}{
	\existe{h}{hogar}{
		h\in th \wedge esCasa(h) \wedge\\ \paraTodo{g}{hogar}{
			(g\in th \wedge esCasa(g)) \implica \#habitaciones(h)\geq \#habitaciones(g)
		} \wedge\\
		\#habitaciones(h) = |res|}
}

\subsection{Ejercicio 3}

\begin{proc}{laCasaEstaQuedandoChica}{\In th: $eph_h$, \In ti: $eph_i$, \Out res: \TLista{\float}}{}
	\pre{ encuestaValida(th, ti) \yLuego (\forall{reg: \ent})(1\leq reg\leq 6 \implicaLuego\#casasPorRegion(th, reg)>0)}
	\post{ |res| = 6 \yLuego
		\bloque{+6em}{
			\paraTodo{i}{\ent}{
				0 \leq i < |res| \implicaLuego ( \bloque{+2em}{
					0\leq res[i]\leq 1 \wedge\\ res[i] = \#casasCriticasPorRegion(th, ti, i + 1) / \#casasPorRegion(th, i + 1)
				})
			}
		}
	}
\end{proc}

\aux{\#casasPorRegion}{th: $eph_h$, r: \ent}{\ent}{
	\sum_{h\in th}\IfThenElse{esCasa(h) \wedge h[@region] = r}{1}{0}
}

\aux{\#casasCriticasPorRegion}{th: $eph_h$, ti: $eph_i$, r: \ent}{\ent}{
	\bloque{+2em}{
		\sum_{h\in th}\IfThenElse{esCasaCritica(h, ti) \wedge h[@region] = r}{1}{0}
	}
}

\pred{esCasaCritica}{h: hogar, ti: $eph_i$}{
	esCasa(h) \wedge esHacinamientoCritico(h, ti) \wedge \neg estaEnGranAglomerado(h)
}

\pred{esHacinamientoCritico}{h: hogar, ti: $eph_i$}{
	\#personasEnHogar(ti, h) / \#dormitorios(h) > 3
}

\pred{estaEnGranAglomerado}{h: hogar}{h[@mas500] = 1}

\subsection{Ejercicio 4}

\begin{proc}{creceElTeleworkingEnCiudadesGrandes}{\In t1h: $eph_h$, \In t1i: $eph_i$, \In t2h: $eph_h$, \In t2i: $eph_i$, \Out res: \bool}{}
	\pre{ (encuestaValida(t1h, t1i) \wedge encuestaValida(t2h, t2i)) \yLuego (
		\bloque{+6em}{
			esHogarConA\tilde noPrevio(head(t1h), head(t2h)) \wedge\\
			hogaresConTrimestresIguales(head(t1h), head(t2h))
		})
	}
	\post{ res = true \Iff teleworkingEnCiudadesGrandes(t1h, t1i) < teleworkingEnCiudadesGrandes(t2h,t2i) }
\end{proc}

\pred{esHogarConAñoPrevio}{h: hogar, g: hogar}{
	h[@hogA\tilde no] < g[@hogA\tilde no]
}

\aux{teleworkingEnCiudadesGrandes}{th: $eph_h$, ti: $eph_i$}{\float}{
	\bloque{+2em}{
		\#personasTrabajandoADistanciaEnCiudadesGrandes(th, ti)/\#personasTrabajandoEnCiudadesGrandes(th, ti)
	}
}

\aux{\#personasTrabajandoADistanciaEnCiudadesGrandes}{th: $eph_h$, ti: $eph_i$}{\ent}{
	\bloque{+2em}{
		\sum_{i\in ti}\IfThenElse{viveEnHogarAptoParaTeleworking(i, th) \wedge trabajaADistancia(i)}{1}{0}
	}
}

\aux{\#personasTrabajandoEnCiudadesGrandes}{th: $eph_h$, ti: $eph_i$}{\ent}{
	\bloque{+2em}{
		\sum_{i\in ti}\IfThenElse{viveEnCiudadGrande(i, th) \wedge trabaja(i)}{1}{0}
	}
}

\pred{viveEnHogarAptoParaTeleworking}{i: individuo, th: $eph_h$}{
	\existe{h}{hogar}{
		h\in th \wedge\\ personaEnHogar(i, h) \wedge h[@mas500] = 1 \wedge
		h[@ii3] = 1 \wedge (h[@iv1] = 1 \vee h[@iv1] = 2)
	}
}

\pred{trabajaADistancia}{i: individuo}{
	i[@pp04g]=6 \wedge trabaja(i)
}

\pred{trabaja}{i: individuo}{
	i[@estado] = 1
}

\pred{viveEnCiudadGrande}{i: individuo, th: $eph_h$}{
	\existe{h}{hogar}{
		h\in th \wedge\\ personaEnHogar(i, h) \wedge h[@mas500] = 1
	}
}

\subsection{Ejercicio 5}

\begin{proc}{costoSubsidioMejora}{\In th: $eph_h$, \In ti: $eph_i$, \In monto: \ent, \Out res: \ent}{}
	\pre{ encuestaValida(th, ti) \wedge monto > 0 }
	\post{ res = \sum_{h\in th}\IfThenElse{esHogarSubsidiado(h, ti)}{monto}{0} }
\end{proc}

\pred{esHogarSubsidiado}{h: hogar, ti: $eph_i$}{
	esCasa(h) \wedge h[@ii7] = 1 \wedge \#dormitorios(h) < \#personasEnHogar(ti,h) - 2
}

\subsection{Ejercicio 6}

\begin{proc}{generarJoin}{\In th: $eph_h$, \In ti: $eph_i$, \Out junta: $joinHI$}{}
	\pre{ encuestaValida(th, ti) }
	\post{
		(\forall hi: hogar\times individuo)(
		\bloque{+6em}{hi\in junta \implica (\bloque{+2em}{
				hi_0\in th \wedge hi_1\in ti \yLuego hi_0[@hogCodusu] = hi_1[@indCodusu]
			})}
	})
\end{proc}

\subsection{Ejercicio 7}

\begin{proc}{ordenarRegionYTipo}{\Inout th: $eph_h$, \Inout ti: $eph_i$}{}
	\pre{ encuestaValida(th, ti) \wedge th = TH \wedge ti = TI }
	\post{
		\bloque{+6em}{
			|th| = |TH| \wedge |ti| = |TI| \yLuego\\
			(\forall h: hogar)(h\in th \Iff h\in TH) \wedge\\
			(\forall i: individuo)(i\in ti \Iff i\in TI) \wedge\\
			hogaresOrdenados(th) \wedge individuosOrdenados(th, ti)
		}
	}
\end{proc}

\pred{hogaresOrdenados}{th: $eph_h$}{
	(\forall i: \ent)
	(0\leq i< |th| -1 \implicaLuego (\bloque{+2em}{
		th[i][@region]\leq th[i+1][@region] \wedge\\
		(th[i][@region] = th[i+1][@region] \implica th[i][@hogCodusu] < th[i+1][@hogCodusu])
	})
}

/* Falta punto 1) */\\
\pred{individuosOrdenados}{th: $eph_h$, ti:
$eph_i$}{
(\forall i: \ent)
(0\leq i< |ti| -1 \implicaLuego
\bloque{+2em}{
ti[i][@indCodusu] = ti[i+1][@indCodusu] \implica
ti[i][@componente]<ti[i+1][@componente]
})
}

\subsection{Ejercicio 8}

\begin{proc}{muestraHomogenea}{\In th: $eph_h$, \In ti: $eph_i$, \Out res: \TLista{hogar}}{}
	\pre{ encuestaValida(th, ti) }
	\post{  \existe{s}{\TLista{hogar}}{
			\bloque{+6em}{
				esHomogenea(s, th, ti) \wedge\\
				(\forall t: \TLista{hogar})(esHomogenea(t, th, ti) \implica |s| \geq |t|) \wedge (
				\bloque{+2em}{
					(
					|s| < 3 \implica res = <>
					) \wedge\\
					(
					|s| \geq 3 \implica res = s
					)
				})
			}}
	}
\end{proc}

\pred{esHomogenea}{s: \TLista{hogar}, th: $eph_h$, ti: $eph_i$}{
	mismaDiferencia(s) \wedge ordenadaPorIngresos(s,ti) \wedge (\forall h: $hogar$)(h\in s \implica h\in th)
}

\pred{mismaDiferencia}{s: \TLista{hogar}}{
	(\forall i: \ent) (1\leq i < |s| - 1 \implicaLuego diferenciaDeIngresos(s[i], s[i-1], ti) = diferenciaDeIngresos(s[i], s[i+1], ti))
}

\pred{ordenadaPorIngresos}{s: \TLista{hogar}, ti: $eph_i$} {
	(\forall i: \ent)(
	0\leq i < |s| - 1 \implicaLuego
	ingresosEnHogar(s[i], ti) \leq ingresosEnHogar(s[i+1], ti)
	)
}

\aux{diferenciaDeIngresos}{a,b: $hogar$, ti: $eph_i$}{\ent}{
	|ingresosEnHogar(a, ti) - ingresosEnHogar(b, ti)|
}

\aux{ingresosEnHogar}{h: $hogar$, ti: $eph_i$}{\ent}{
	\bloque{+2em}{
		\sum_{i=0}^{|ti|-1}\IfThenElse{personaEnHogar(ti[i], h) \wedge ti[i][@p47T] \neq -1}{ti[i][@p47T]}{0}
	}
}

\subsection{Ejercicio 9}

\begin{proc}{corregirRegion}{\Inout th: $eph_h$, \In ti: $eph_i$}{}
	\pre{ encuestaValida(th, ti) \wedge th = TH }
	\post{ |th|=|TH| \yLuego (\forall i: \ent)(
	0\leq i < |th| \implicaLuego \bloque{+6em}{ (TH[i][@region]=1 \implica
	\bloque{+2em}{ th[i][@region]=5 \wedge\\
	th[i][@hogCodusu]=TH[i][@hogCodusu] \wedge\\
	th[i][@hogA\tilde no]=TH[i][@hogA\tilde no] \wedge\\
	th[i][@hogTrimestre]=TH[i][@hogTrimestre] \wedge\\
	th[i][@hogLatitud]=TH[i][@hogLatitud] \wedge\\
	th[i][@hogLongitud]=TH[i][@hogLongitud] \wedge\\
	th[i][@ii7]=TH[i][@ii7] \wedge\\
	th[i][@mas500]=TH[i][@mas500] \wedge\\
	th[i][iv1]=TH[i][iv1] \wedge\\
	th[i][@iv2]=TH[i][@iv2] \wedge\\
	th[i][@ii2]=TH[i][@ii2] \wedge\\
	th[i][@ii3]=TH[i][@ii3] } )\wedge\\ (TH[i][@region]\neq1
	\implica th[i]=TH[i]) } ) }
\end{proc}

\subsection{Ejercicio 10}

\begin{proc}{histogramaDeAnillosConcentricos}{\In th: $eph_h$, \In centro: \ent\times\ent, \In distancias: \TLista{\ent}, \Out result: \TLista{\ent}}{}
	\pre{\bloque{+6em}{|distancias| > 0 \wedge\\
			(\forall i: \ent)(0\leq i < |distancias| - 1 \implicaLuego 0 < distancias[i] < distancias[i+1])}}
	\post{ |result| = |distancias| \yLuego
		\bloque{+6em}{
			\paraTodo{i}{\ent}{
				0\leq i < |result| \implicaLuego (\bloque{+2em}{
					(i=0\implica res[i] = hogaresEnRadio(th, centro, -1, distancias[0]))\wedge\\
					(i \neq 0\implica res[i] = hogaresEnRadio(th, centro, distancias[i-1], distancias[i]))} )
			}
		}
	}
\end{proc}

\aux{hogaresEnRadio}{th: $eph_h$, centro: \ent\times\ent,
	desde:\ent ,hasta:\ent}{\ent}{
	\bloque{+2em}{\sum_{j=0}^{|th|-1}\IfThenElse{desde<distanciaEuclidiana(cent ro,(th[j][@hogLatitud], th[j][@hogLongitud])) \leq hasta}{1}{0}}
}

\aux{distanciaEuclidiana}{c1,c2: \ent\times\ent}{\float}{
	\sqrt{(c2_0-c1_0)^2+(c2_1-c1_1)^2}
}

\subsection{Ejercicio 11}

\begin{proc}{quitarIndividuos}{\Inout th: $eph_h$, \Inout ti: $eph_i$, \In busqueda: \TLista{ItemIndividuo\times dato}, \Out result: $eph_h$\times$eph_i$}{}
	\pre{ encuestaValida(th, ti) \wedge busquedaValida(busqueda) \wedge th = TH \wedge ti = TI }
	\post{ 0\leq |th|\leq |TH| \wedge 0\leq |ti|\leq |TI| \wedge |result| = |TI| - |ti| \yLuego
		\bloque{+6em}{
			\paraTodo{h}{hogar}{(h \in TH\wedge hogarLlenoDeCoincidentes(TI, h, busqueda))\Iff(h\in result_0 \wedge h \notin th)} \wedge \\
			\paraTodo{i}{individuo}{(i \in TI\wedge coincideConTerminos(i, busqueda))\Iff(i\in result_1 \wedge i \notin ti)}
		}
	}
\end{proc}

\aux{\#personasEnHogarCoincidentes}{ti: $eph_i$, h: $hogar$, busqueda: \TLista{ItemIndividuo\times dato}}{\ent}{
	\bloque{+2em}{\sum_{j=0}^{|ti|-1} \IfThenElse{personaEnHogar(ti[j], h)\wedge coincideConTerminos(ti[j], busqueda)}{1}{0}}
}

\pred{hogarLlenoDeCoincidentes}{ti: $eph_i$, h: $hogar$, busqueda: \TLista{ItemIndividuo\times dato}}{
	\#personasEnHogar(ti,h) = \#personasEnHogarCoincidentes(ti,h,busqueda)
}

\pred{busquedaValida}{busqueda: \TLista{ItemIndividuo\times dato}}{
	\paraTodo{b}{ItemIndividuo\times dato}{b\in busqueda \implica def(ord(b_0))} \yLuego\\
	\neg\existe{i,j}{\ent}{0\leq i,j < |busqueda| \wedge i\neq j \yLuego
		ord(busqueda[i]_0)=ord(busqueda[j]_0)}{}
}

\pred{coincideConTerminos}{i : individuo, busqueda: \TLista{ItemIndividuo\times dato} }{
	\paraTodo{t}{\ent}{0\leq t < |busqueda| \implicaLuego i[ord(busqueda[t]_0)] = busqueda[t]_1}
}

\end{document}