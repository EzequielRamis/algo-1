\documentclass[11pt]{article}

\input{Algo1Macros.tex}
\usepackage{caratula}

\begin{document}
%Caratula
\titulo{Titulo}
\subtitulo{subtitulo}
\fecha{fecha}
\materia{materia}
\integrante{Fulanito, Cosme}{000/17}{fulanito.cosme@gmail.com}
\integrante{Bond, James}{007/17}{bond.james@gmail.com}
\maketitle
%Creación de indice
\tableofcontents
\newpage

El factorial de un entero positivo $n$ se define como:
$n! = \prod_{i=1}^{n}i$

El factorial de $5$ es: %
$5! = \prod_{i=1}^{5}i=1\times2\times3\times4\times5=120$

\section{Especificación}
\begin{proc}{factorial}{\In n: \ent, \Out result: \ent}{}
	\pre{n\geq0}
	\post{(n=0\implica result=1)\wedge(n>0\implica result=\prod_{k=1}^nk)}
\end{proc}

\pred{esImpar}{s: \TLista{\ent}}
{(\forall i: \ent)(0\leq i<\longitud{s}\implicaLuego esPrimo(s[i]))}

\pred{alMenosUnPrimo}{s: \TLista{\ent}}
{(\exists i: \ent)(0\leq i<\longitud{s}\yLuego esPrimo(s[i]))}

\aux{sumaPrimos}{s: \TLista{\ent}}{\ent}
{\sum_{i=0}^{\longitud{s}-1}\IfThenElse{esPrimo(s[i])}{s[i]}{0}}

$$
	\hat{R}(\hat{f},\bar{D}_n^m)=\frac{1}{|\bar{D}_n^m|}
	\sum_{i:(X_i,Y_i)\in \bar{D}_n^m}(Y_i-\hat{f}_m,\hat{D}_n^m)
$$

😎😎😎

\end{document}